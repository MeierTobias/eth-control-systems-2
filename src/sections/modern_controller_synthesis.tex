\section{Modern Controller Synthesis}
To unify the controller synthesis the following standard form has been established.


\setlength{\oldtabcolsep}{\tabcolsep}\setlength\tabcolsep{2pt}

\begin{tabularx}{\linewidth}{@{}ll@{}}
    \includegraphics[width=0.5\linewidth, align=c]{controller_synthesis_paradigm.png}
     &
    {\begin{tabularx}{\linewidth}{@{}ll@{}}
                 $u$: & control input            \\
                 $y$: & measured output          \\
                 $r$: & reference signal         \\
                 $w$: & disturbance              \\
                 $z$: & performance output       \\
                 $q$: & uncertainty block input  \\
                 $v$: & uncertainty block output
             \end{tabularx}
        }
\end{tabularx}

\setlength{\tabcolsep}{\oldtabcolsep}


\newpar{}
The objective is to stabilize the system in presence of $\boldsymbol{\Delta}$ and minimize the performance outputs $z$, given the exogenous reference and disturbances.

\subsection{Frequency Weights}

To guide the optimization process the performance outputs $z_i$ are weighted with frequency dependent weighting functions $W_i$.

The most common performance outputs are
\begin{center}
    \includegraphics[width=0.9\linewidth]{frequency_weights.png}
\end{center}
with the known transfer functions
\begin{align*}
    L(s) & =P(s)C(s)                     & \text{ loop TF}            \\
    S(s) & =G_{er}(s) = {(I+L(s))}^{-1}  & \text{ sensitivity}        \\
    T(s) & = G_{yr}(s)                   &                            \\
         & ={(I+L(s))}^{-1}L(s) = I-S(s) & \text{ comp.\ sensitivity}
\end{align*}

\newpar{}
\ptitle{Tracking Error}

Is given by
\begin{equation*}
    z_1=W_1 G_{er} = W_1 S(s)
\end{equation*}
\begin{itemize}
    \item If $W_1(s)$ is chosen large at low frequencies $S(s)$ must be small in order to minimize $z_1$. In other words, the tracking error must be small at low frequencies.
    \item This is the same as requiring $\sigma_{\min}[L(s)] \gg |W_1(s)|$, which corresponds to the low-frequency ``Bode obstacle'' in the SISO case.
\end{itemize}

\newpar{}
\ptitle{Control Effort}

Is given by
\begin{equation*}
    z_2=W_2 G_{ur}(s) = W_2 C(s) S(s)
\end{equation*}
\begin{itemize}
    \item Useful to limit the maximum control effort (large $W_2$ yields small $u$)
\end{itemize}

\newpar{}
\ptitle{Noise Rejection}

Is given by
\begin{equation*}
    z_3= W_3 G_{yn} =-W_3 T(s)
\end{equation*}
\begin{itemize}
    \item If $W_3$ is chosen very large for high frequencies, the complementary sensitivity must be very small at high frequencies.
    \item This is the same as requiring $\sigma_{\max}[L(s)] \ll |W_3(s)|$, which corresponds to the high-frequency ``Bode obstacle'' in the SISO case.
\end{itemize}

\newpar{}
\ptitle{Stability Robustness}

For stability robustness in the presence of uncertainty $\Delta$ the same weighting method can be applied.
\begin{equation*}
    z_\Delta = \Delta(s)W_\Delta(s)
\end{equation*}
with
\begin{gather*}
    \|\Delta(s)\|_{\mathcal{H}_\infty} < 1 \\
    \|M(s)\|_{\mathcal{H}_\infty} < 1
\end{gather*}
where $M(s)$ is the TF from the output of $\Delta$ to its input.

\subsubsection{First Order Weights}
\textbf{MATLAB}: \texttt{makeweight}
\newpar{}
\ptitle{Lowpass}
\begin{equation*}
    W(s) = \frac{M}{\frac{Ms}{\omega}+1}
\end{equation*}

\ptitle{Highpass}
\begin{equation*}
    W(s) = \frac{Ms}{s+M\omega}
\end{equation*}

\subsection{State Space Representation}
To synthesize a controller an assembled state space model of the generalized system ($G$) can be constructed. This yields a similar approach as the transfer function-based one before.

\ptitle{Plant}
\begin{gather*}
    P : \left[
        \begin{array}{c|c} % ChkTex -2
            A_p & B_p \\
            \hline % ChkTex -2
            C_p & D_p \\
        \end{array}
        \right]                            \\
    P(s) = {C_p(sI-A_p)}^{-1}B_p+D_p
\end{gather*}
\textbf{MATLAB}: \texttt{P=ss(Ap, Bp, Cp, Dp)}

\ptitle{Frequency Weights}

For each weight we get
\begin{gather*}
    W : \left[
        \begin{array}{c|c} % ChkTex -2
            -p   & 1 \\
            \hline % ChkTex -2
            r-qp & q
        \end{array}
        \right]                   \\
    W(s) = \frac{qs+r}{s+p}
\end{gather*}

\ptitle{Generalized System}

The generalized system is obtained by stacking all the state-space models:
\begin{gather*}
    G : \left[
        \begin{array}{c|c c} % ChkTex -2
            A   & B_w    & B_u    \\
            \hline % ChkTex -2
            C_z & D_{zw} & D_{zu} \\
            C_y & D_{yw} & D_{yu}
        \end{array}
        \right]                        \\
    G(s) = \begin{bmatrix}
        G_{zw}(s) & G_{zu}(s) \\
        G_{yw}(s) & G_{yu}(s) \\
    \end{bmatrix}
\end{gather*}
The matrix $A$ contains both the dynamics of the plant and the frequency weights.\\
\textbf{MATLAB}: \texttt{augw(P, W1, W2, W3)}

\newpar{}
\ptitle{Controller}
\begin{gather*}
    K : \left[
        \begin{array}{c|c} % ChkTex -2
            A_c & B_c \\
            \hline % ChkTex -2
            C_c & D_c \\
        \end{array}
        \right]                            \\
    K(s) = {C_c(sI-A_c)}^{-1}B_c+D_c
\end{gather*}

As $u(s)=K(s)y(s)$ the closed loop TF from the exogenous inputs $w$ to the performance outputs $z$ is given by the \textit{Linear Fractional Transformation}:
\begin{equation*}
    F(s) = G_{zw}(s) + G_{zu}(s)K(s){(I-G_{yu}(s))}^{-1}G_{yw}(s)
\end{equation*}

\textbf{MATLAB}: \texttt{F = lft(G, K)}

\subsubsection{Controller Synthesis}
Given $F$ one can use various control synthesis techniques, e.g.\
\newpar{}
\ptitle{$\mathcal{H}_2$ Design}

Minimize
\begin{equation*}
    \|F\|_{2}:={\left(\frac{1}{2\pi}\int_{-\infty}^{+\infty}\mathrm{Tr}[{F(j\omega)}^{*}F(j\omega)]\mathrm{~d}\omega\right)}^{1/2}
\end{equation*}
i.e.\
\begin{equation*}
    \int_0^{+\infty}\|f(t)\|_2^2dt
\end{equation*}

\newpar{}
\ptitle{$\mathcal{H}_\infty$ Design}

Minimize
\begin{equation*}
    \|F\|_\infty:=\sup_\omega\sigma_{\max}[F(j\omega)]
\end{equation*}
i.e.\
\begin{equation*}
    \int_0^{+\infty}\|z(t)\|_2^2dt
\end{equation*}
for unit-energy inputs i.e.\ $\int_0^{+\infty}\|w(t)\|_2^2 dt=1$

\begin{examplesection}[Example Assembly]
    For the system
    \begin{center}
        \begin{tikzpicture}[auto, node distance=1.5cm,>=latex']
    \node [input, name=uinput]          (uinput)        {};
    \node [tmp, right of=uinput]        (ut)            {};
    \node [block, right of=ut]          (P)             {$P(s)$};
    \node [block, above of=P]           (W)             {$W(s)$};
    \node [output, right of=P]          (y)             {};
    \node [output, right of=W]          (z)             {};

    \draw [->] (rinput) -- node[pos=0.2]{$u$} (P);
    \draw [->] (ut) |- node{} (W);
    \draw [->] (P) -- node{$y$} (y);
    \draw [->] (W) -- node{$z$} (z);
\end{tikzpicture}
    \end{center}
    the generalized system $G(s)$ has to be calculated
    \begin{equation*}
        \begin{bmatrix}
            y \\
            z
        \end{bmatrix}
        = \underbrace{\begin{bmatrix}
                G_{yu}(s) \\
                G_{zu}(s)
            \end{bmatrix}}_{G(s)} u
    \end{equation*}
    The combined states evolve with
    \begin{equation*}
        \begin{bmatrix}
            \dot{x}_p \\
            \dot{x}_w
        \end{bmatrix}
        = \underbrace{\begin{bmatrix}
                A_p & 0   \\
                0   & A_w
            \end{bmatrix}}_{A_G}
        \begin{bmatrix}
            x_p \\
            x_w
        \end{bmatrix}
        + \underbrace{\begin{bmatrix}
                B_p \\
                B_w
            \end{bmatrix}}_{B_G} \: u
    \end{equation*}

    and the combined output is given by

    \begin{equation*}
        \begin{bmatrix}
            y \\
            z
        \end{bmatrix}
        = \underbrace{\begin{bmatrix}
                C_p & 0   \\
                0   & C_w
            \end{bmatrix}}_{C_G}
        \begin{bmatrix}
            x_p \\
            x_w
        \end{bmatrix}
        +
        \underbrace{\begin{bmatrix}
                D_p \\
                D_w
            \end{bmatrix}}_{D_G}
        \: u
    \end{equation*}
    The generalized system representation is then given by
    \begin{equation*}
        G = \left[
            \begin{array}{c|c} % ChkTex -2
                A_G & B_G \\
                \hline % ChkTex -2
                C_G & D_G
            \end{array}
            \right]
        =
        \left[
            \begin{array}{cc|c} % ChkTex -2
                A_p & 0   & B_p \\
                0   & A_w & B_w \\
                \hline % ChkTex -2
                C_p & 0   & D_p \\
                0   & C_w & D_w
            \end{array}
            \right]
    \end{equation*}

\end{examplesection}

\subsection{Youla's Q Parameterization}
% TODO: 
%-Structure of this section could be optimized
%-Remark on SISO/MIMO
One can get all stabilizing controllers for a given plant as a function of a single stable transfer function Q(s). This is called the Youla parameterization (Q-parameterization).
\begin{itemize}
    \item Extension of a full state feedback controller (including observer) with a system \textcolor{purple}{$Q$} that takes the innovation as an input.
    \item Valid in a very general setting, including MIMO
    \item $Q$ needs to be \textbf{stable} and has to consist of a \textbf{proper rational} transfer function. Reminder:
          \begin{itemize}
              \item Proper: $\deg(Den)\le \deg(Num)$
              \item Rational: fraction of two polynomials
          \end{itemize}
    \item Note that positive feedback is used (common in modern control).
\end{itemize}

\begin{center}
    \includegraphics[width=0.8\linewidth]{youlas_Q_parameterization.png}
\end{center}
\newpar{}
Assuming a controllable and observable plant, the closed-loop dynamics are given by
\begin{equation*}
    \begin{bmatrix}
        \dot{x}   \\
        \dot{x}_Q \\
        \dot{\eta}
    \end{bmatrix}
    =
    \begin{bmatrix}
        A+BK & BC_Q & -BK   \\
        0    & A_Q  & B_Q C \\
        0    & 0    & A+LC
    \end{bmatrix}
    \begin{bmatrix}
        x   \\
        x_Q \\
        \eta
    \end{bmatrix}
    +
    \begin{bmatrix}
        I & 0 \\
        0 & C \\
        0 & C
    \end{bmatrix}
    \begin{bmatrix}
        r \\
        d
    \end{bmatrix}
\end{equation*}
The closed-loop poles are the union of the full-state feedback poles $A+BK$, the observer poles $A+LC$ and the poles of $Q(s)$.

\subsubsection{SISO Controller Design}
\newpar{}
The TF from $y$ to $u$ hence the TF of the controller $C(s)$ can be written as
\begin{equation*}
    C(s) = -{(X_0(s)+Q(s)N(s))}^{-1}(Y_0(s)-Q(s)D(s))
\end{equation*}
where
\begin{equation*}
    C_0(s)=-{X_0(s)}^{-1}Y_0(s)
\end{equation*}
is the TF when $Q(s)=0$ with controller gain $K_0$ and observer gain $L_0$
\noindent\begin{align*}
    X_0(s) & = -K_0{(sI-A-L_0C)}^{-1}B   \\
    Y_0(s) & = -K_0{(sI-A-L_0C)}^{-1}L_0
\end{align*}
and $N$, $D$ are the nominator and denominator of the plant
\begin{align*}
    N(s) & =C{(sI-A-L_0C)}^{-1}B    \\
    D(s) & = -C(sI-A-L_0-C^{-1}L_0)
\end{align*}
where $L_0$ is an observer gain that would stabilize the error dynamics.

\paragraph{Finding a Stabilizing Controller}
It can be shown that the interconnection of the system is stable if the \textbf{Bezout identity} is fulfilled
\begin{equation*}
    D(s)X(s)-N(s)Y(s)=1
\end{equation*}
and that all feedback stabilizing controllers for a \textbf{SISO}(!) system $P$ are given by
\begin{equation*}
    C(s) = \frac{Y_0(s) -D(s)Q(s)}{X_0(s)-N(s)Q(s)}
\end{equation*}
One can then
\begin{enumerate}
    \item Find a default ``dummy'' controller $C_0(s)$.
    \item Plug it into $C(s)$.
    \item Use optimization algorithms to find a better controller $C(s)$ by systematically iterating through many \textbf{stable} $Q$ functions.
\end{enumerate}

\subsubsection{(Complementary) Sensitivity}
The sensitivity function is given by
\begin{equation*}
    S(s) = D(s)(X_0(s)-N(S)Q(s))
\end{equation*}
and the complementary sensitivity function by
\begin{equation*}
    T(s) = -N(s)(Y_0(s)+D(s)Q(s))
\end{equation*}