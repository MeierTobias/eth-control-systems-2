\section{LQR}
More details on the derivation of LQR can be found under~\ref{h2_synth} $\mathcal{H}_2$ Synthesis.
\subsection{Cost Functional}
LQR can be used to systematically optimize the \textbf{static gain vector K} based on \textit{tracking errors} and \textit{control effort}.

For the reference input $r(t) = 0$ the cost functional $J$ is defined as
\noindent\begin{equation*}
    J(\mathbf{x,t})=\int_0^{+\infty} \Bigl[\underbrace{{\mathbf{x}(t)}^{\mathsf{T}}\mathbf{Qx}(t)}_{\textsf{tracking error}} + \underbrace{{\mathbf{u}(t)}^{\mathsf{T}} \mathbf{Ru}(t)}_{\textsf{control effort}} \Bigr]dt
\end{equation*}
\textbf{Remarks:}
\begin{itemize}
    \item $\mathbf{Q}$
          \begin{itemize}
              \item is symmetric i.e. $\mathbf{Q}=\mathbf{Q}^{\mathsf{T}}$
              \item is positive semidefinite i.e. $\mathbf{x}^{\mathsf{T}} \mathbf{Qx} \geq 0$ or $\mathrm{eig}(\mathbf{Q}) \geq 0$
              % TODO: Why must it be sqrt (below)? Compare with LQE technical conditions
              \item must give positive weights to unstable modes to penalize them in the cost function (the pair $\mathbf{A}, \sqrt{\mathbf{Q}}$ is detectable)
          \end{itemize}
    \item $\mathbf{R}$ 
    \begin{itemize}
        \item must be positive definite i.e. $\mathbf{u}^{\mathsf{T}} \mathbf{Ru} >0\; \forall \mathbf{u}\neq \mathbf{0}$ or $\mathrm{eig}(\mathbf{R}) > 0$ (otherwise not every control effort is penalized).
        \item is symmetric (see derivation in H2-Synthesis Section~\ref{h2_synth})
    \end{itemize}
    \item The pair $\mathbf{A,B}$ must be stabilizable
    \item In general, there is an additional cross-coupling term $\mathbf{x}^{\mathsf{T}}\mathbf{Nu}$ in $J$ which is often neglected i.e.\ equal to $0$
\end{itemize}

\subsubsection{Bryson's Rule}
Bryson's rule can be used as a \textit{baseline guess} for $\mathbf{Q,R}$:
\noindent\begin{align*}
    \mathbf{Q} =
     & \begin{bmatrix}\frac{q_1^2}{|x_{1,\max}|^2} &                              &  &        &                              & \\
                                            & \frac{q_2^2}{|x_{2,\max}|^2} &  &        &                              & \\
                                            &                              &  & \ddots &                              & \\
                                            &                              &  &        & \frac{q_n^2}{|x_{n,\max}|^2}
       \end{bmatrix} \\
    \mathbf{R} = \rho
     & \begin{bmatrix}\frac{r_1^2}{|u_{1,\max}|^2} &                              &  &        &                              & \\
                                            & \frac{r_2^2}{|u_{2,\max}|^2} &  &        &                              & \\
                                            &                              &  & \ddots &                              & \\
                                            &                              &  &        & \frac{r_n^2}{|u_{n,\max}|^2}
       \end{bmatrix}
\end{align*}
where
\noindent\begin{equation*}
    \sum_{i=1}^n q_i^2=1\text{ and }\sum_{i=1}^m r_i^2=1
\end{equation*}
and $x_{i,\max}$, $u_{j,\max}$ are the \textbf{maximum deviations} that we are willing to tolerate for the $i$-th state and $j$-th input, respectively.

\subsection{Continuous Time}
First, check $\mathbf{A,B}$ for reachability!

In continuous time, first the \textbf{symmetric, positive definite} solution
\begin{equation*}
    \mathbf{P}=\begin{bmatrix} p_{1,1} & \dots  & p_{1,n} \\
                        & \ddots &         \\
                p_{1,n} & \dots  & p_{n,n}
    \end{bmatrix}
\end{equation*}
with
\begin{align*}
    p_{1,1}                   & >0 \\
    p_{1,1}p_{2,2} -p_{1,2}^2 & >0 \\
    \vdots
\end{align*}
of the \textit{continuous algebraic Riccati equation} (\textbf{\textit{CARE}}) has to be found
\noindent\begin{equation*}
    \mathbf{A}^{\mathsf{T}} \mathbf{P}+\mathbf{PA}+\mathbf{Q}-\mathbf{PBR}^{-1}\mathbf{B}^{\mathsf{T}} \mathbf{P}=0
\end{equation*}
then the \textit{static gain matrix} $\mathbf{K}$ can be obtained by setting $\mathbf{u} = -\mathbf{Kx}$:
\noindent\begin{equation*}
    \mathbf{K}=\mathbf{R}^{-1}\mathbf{B}^{\mathsf{T}} \mathbf{P}
\end{equation*}

\textbf{Remarks:}

\begin{itemize}
    \item LQR applies directly to MIMO-systems
\end{itemize}

\newpar{}
\textbf{\code{Matlab:}} \code{[K,P]=lqr(A,B,Q,R)}


\subsection{Discrete Time}
In discrete time, the cost functional is
\noindent\begin{equation*}
    J(\mathbf{x,u})=\sum_{k=0}^{+\infty}\left({\mathbf{x}[k]}^{\mathsf{T}} \mathbf{Qx}[k]+{\mathbf{u}[k]}^{\mathsf{T}} \mathbf{Ru}[k]\right),
\end{equation*}
the \textbf{symmetric, positive definite} solution $\mathbf{P}$ to the \textit{discrete algebraic Riccati equation} (\textbf{\textit{DARE}})
\noindent\begin{equation*}
    \mathbf{P}=\mathbf{A}^{\mathsf{T}} \mathbf{PA}-(\mathbf{A}^{\mathsf{T}} \mathbf{PB}){(\mathbf{B}^{\prime}\mathbf{PB}+\mathbf{R})}^{-1}(\mathbf{B}^{\mathsf{T}} \mathbf{PA})+\mathbf{Q}
\end{equation*}
and the \textit{static gain matrix} obtained with $\mathbf{u} = -\mathbf{Kx}$ is
\noindent\begin{equation*}
    \mathbf{K}={(\mathbf{R}+\mathbf{B}^{\mathsf{T}} \mathbf{PB})}^{-1}\mathbf{B}^{\mathsf{T}} \mathbf{PA}
\end{equation*}

\textbf{\code{Matlab:}} \code{[K\_d,P\_d]=dlqr(A,B,Q,R)}

\subsection{LQR Servo: Integrator}
To compensate model errors or disturbances an integrator can be added.
This is simply done by adding the integral of the error to the state space model
\noindent\begin{align*}
    \dot{x}_I & =-y+r =  -\mathbf{Cx} - \mathbf{Du} + r \\
    \frac d{dt}
    \begin{bmatrix}
        \mathbf{x} \\
        x_I
    \end{bmatrix}
              & =\begin{bmatrix}
                     \mathbf{A}  & \mathbf{0} \\
                     -\mathbf{C} & 0
                 \end{bmatrix}
    \begin{bmatrix}
        \mathbf{x} \\
        x_I
    \end{bmatrix}
    +\begin{bmatrix}
         \mathbf{B} \\
         \mathbf{D}
     \end{bmatrix}
    \mathbf{u}+
    \begin{bmatrix}
        0 \\
        1
    \end{bmatrix}
    r                                                   \\
              & =\begin{bmatrix}
                     \mathbf{A}  & \mathbf{0} \\
                     -\mathbf{C} & 0
                 \end{bmatrix}
    \begin{bmatrix}
        \mathbf{x} \\
        x_I
    \end{bmatrix}
    +\begin{bmatrix}
         \mathbf{B} & \mathbf{0} \\
         \mathbf{D} & 1
     \end{bmatrix}
    \begin{bmatrix}
        \mathbf{u} \\
        r
    \end{bmatrix},
\end{align*}
then apply LQR to the modified state space model which yields a controller
\begin{align*}
    \mathbf{K}' & =\begin{bmatrix}
                       \mathbf{K} & K_I
                   \end{bmatrix}                          \\
    \mathbf{u}  & =-\mathbf{K}'\begin{bmatrix}
                                   \mathbf{x} \\
                                   x_I
                               \end{bmatrix} + \mathbf{K'S}r
\end{align*}

\subsection{Symmetric Root Locus}
The symmetric root locus can be used to understand the influence of $\rho$ (penalizes control effort) on the poles of a closed loop LQR feedback system.
By using $\mathbf{Q}=\mathbf{C}^{\mathsf{T}} \mathbf{C}$ and $\mathbf{R}=\rho \mathbf{I}$, the symmetric root locus condition simplifies to
\noindent\begin{align*}
    \rho \mathbf{I}+\mathbf{G}(s)\mathbf{G}(-s)                  & =\mathbf{0} \qquad \Leftrightarrow \\
    \rho \mathbf{D}(s)\mathbf{D}(-s)+\mathbf{N}(s)\mathbf{N}(-s) & =\mathbf{0}
\end{align*}
where $\mathbf{G}(-s)$ corresponds to flipping signs in all the odd coefficients.

\newpar{}
\ptitle{Symmetric Root Locus Rules}

\begin{itemize}
    \item $2n$ branches, where $n$ is the size of $\mathbf{A}$
    \begin{itemize}
        \item like in standard RL, every branch starts in an OL pole
        \item but asymptotic behavior different from standard RL
    \end{itemize}
    \item symmetric to the real and imaginary axis
    \item LQR closed-loop poles are all in the LHP
    \item $\rho \rightarrow \infty$ (expensive control):\newline
          CL poles approach stable OL poles and the mirror-images of the unstable OL poles (all on LHP)
    \item $\rho \rightarrow 0$ (cheap control):\newline
          CL poles approach MP OL zeros and the mirror-images of the NMP OL zeros or go to infinity along the LHP asymptotes (all on LHP)
    \item note that for $\rho \rightarrow \infty$ there are no asymptotes to $\infty$ but only convergence towards (mirror) OL poles
    \begin{itemize}
        \item for $\rho \rightarrow 0$ there is also convergence to $\infty$
    \end{itemize}
\end{itemize}

\begin{examplesection}[Example: Symmetric Root Locus]
    Consider the LTI system
    \noindent\begin{equation*}
        G = \frac{s-1}{(s-0.5)(s^2+2s+s)}
    \end{equation*}
    the corresponding root-loci are
    \begin{center}
        \includegraphics[width = \linewidth]{symm_rlocus.png}
    \end{center}
    Here the open-loop system features both a non-minimum-phase zero and an unstable pole.
    When looking at the left half of the \textit{symmetric root locus}, the mirrored NMP zero and mirrored unstable pole stabilize the system such that
    for any $\rho$ the closed loop poles will stay on the LHP.
\end{examplesection}

\subsection{LQR Margins: Kalman Frequency Domain Equality}
Considering a SISO LTI system
\noindent\begin{align*}
    \dot{\mathbf{x}} & = \mathbf{Ax} + \mathbf{Bu}                                 \\
    y                & = \mathbf{Cx}                                               \\
    \mathbf{Q}       & = \mathbf{C}^{\mathsf{T}} \mathbf{C}, \quad \mathbf{R}=\rho
\end{align*}
and
\noindent\begin{align*}
    G(s) & = \mathbf{C}{(s \mathbf{I}-\mathbf{A})}^{-1}\mathbf{B} & r\to y \\
    L(s) & = \mathbf{K}{(s \mathbf{I}-\mathbf{A})}^{-1}\mathbf{B} & r\to u
\end{align*}
the \textit{Kalman Frequency Domain Equality}
\noindent\begin{equation*}
    (1+L(-s))(1+L(s)) = 1+\frac{1}{\rho}G(-s)G(s)
\end{equation*}
holds.

\subsubsection{Margins}
In the frequency domain ($s=j\omega$)
\noindent\begin{equation*}
    \left|1+L(j\omega)\right|^2=1+\frac1\rho\left|G(j\omega)\right|^2\geq1
\end{equation*}
In the Nyquist plot, the transfer function using the LQR gain is always outside of a unit circle centered at -1 i.e.
\begin{itemize}
    \item The phase margin is at least 60° because the Nyquist curve cannot enter a unit circe centered at $-1$ %(property of LQR)
    \item Be aware that LQR synthesizes an optimal controller for the \textbf{given model}. However the model is never perfect and the margins to the real system can be smaller due to uncertainties.
    \item The gain margin is ($\frac{1}{2},\infty$) because the loop transfer function (Nyquist curve) will never cross the real axis between ($-2, 0$) (unit circle argument again)\newline $\to -\frac{1}{k} = (\frac{1}{2}, \infty)$
\end{itemize}

