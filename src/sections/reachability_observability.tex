\section{Reachability and Observability}
\noindent\begin{equation*}
    \mathcal{R} \subseteq \mathcal{C}, \quad \mathcal{R} \subset\mathcal{S}, \quad \mathcal{O} \subset \mathcal{D}
\end{equation*}
% Please ignore the mess I've created here, I don't feel comfortable either.
\begin{center}
    \begin{tikzcd}[ampersand replacement=\&]
        \substack{\text{unstable} \\ \text{modes} \\ \in{} \mathcal{R} \\ \in{} \mathcal{O}}      \&[2em] \substack{\text{all} \\ \text{modes}} \&[-3em] \\[-2em]
        \& \mathcal{C} \arrow[d, teal, xshift=-0.5ex,
        "{\color{teal}\begin{matrix}                        %chktex 18
            \mathsf{CT}: & \text{always}                \\
            \mathsf{DT}: & \mathbf{A}_d \text{ invert.}
        \end{matrix}}
        "'                                                  %chktex 18
        ] \& {\footnotesize\begin{cases}
                    \mathsf{CT:} & \mathbf{x}_0 = 0 \to~\mathbf{x_f} \\
                    \mathsf{DT:} & \mathbf{x}_0 = \mathbf{x_f} \to~0
                \end{cases}}
        \\[2em]
        \mathcal{S}                                     \& \mathcal{R} \arrow[l]\arrow[u, xshift=0.5ex] \&
        {\footnotesize\begin{cases}
                    \mathbf{x}_0 = 0 \to~\mathbf{x_f}
                \end{cases}}
        \\
        \mathcal{D}                                     \& \mathcal{O} \arrow[l]    \&
        {\footnotesize\begin{cases}
                    \mathbf{x}_0 \gets~\mathbf{u},\mathbf{y}
                \end{cases}}
    \end{tikzcd}
\end{center}

\begin{itemize}
    \item If a transfer function does not contain any pole-zero-cancellations, it is observable, controllable and reachable.
    \item Duality: $\mathbf{A},\mathbf{B}$ are reachable if and only if $\mathbf{A^T}, \mathbf{B^T}$ are observable.
    \item Structural properties of a system (\textbf{not} affected by similarity transforms or even feedback control!)
\end{itemize}

\subsection{Reachability/Controllability}

An LTI system is reachable if there exists a control signal that takes the system from an initial condition $\mathbf{x_0} = \mathbf{0}$ to any desired final state (vector) $\mathbf{x_f}$, in finite time. This means that each state can be influenced by the input.
\noindent\begin{align*}
    \mathbf{R} :=   & \begin{bmatrix}
                          \mathbf{B} & \mathbf{AB} & \mathbf{A}^2\mathbf{B} & \mathbf{A}^3\mathbf{B} & \cdots & \mathbf{A}^{n-1}\mathbf{B}
                      \end{bmatrix} \\
    \mathbf{x}[n] = & \; \mathbf{RU},\quad \mathbf{U}={\bigl[\mathbf{u}[0],\ldots, \mathbf{u}[n]\bigr]}^{\mathsf{T}}
\end{align*}
Mathematically expressed, a system is reachable if and only if the reachability matrix $\mathbf{R}$ has \textbf{full row rank} $n$.
\begin{align*}
    \text{rank}(\mathbf{R}) & = n                            \\
    \mathbf{x}    & \in \mathrm{Range}(\mathbf{R})
\end{align*}

\newpar{}
\ptitle{Remarks}
\begin{itemize}
    \item If the system is reachable, then $\mathbf{x}[n] = \mathbf{RU}$ can be solved for $\mathbf{U}$ which represents the desired sequence of input signals.
    \item Note that the condition arises from the Cayley Hamilton Theorem.
    \item For \textbf{continuous-time systems} controllability and reachability are the same.
\end{itemize}

\subsubsection{SISO diagonal systems}

The $i$-th mode of a system in diagonal form is reachable only if $\tilde{b}_i \neq 0$ because
\begin{equation*}
    \dot{\tilde{x}}_i =\lambda_i\tilde{x}_i+\tilde{b}_i u
\end{equation*}
This is a necessary but not sufficient condition for reachability of the whole system.

\subsubsection{DT Systems: Controllability}

\begin{itemize}
    \item A DT LTI system is controllable if, for any initial condition $\mathbf{x}_0$, there exists a control input that brings the state $x$ to $0$ in finite time (Note: For Reachability one has the ``opposite'' condition).
    \item Reachability always implies controllability and uncontrollable systems are never reachable
    \item Controllability only implies reachability if $\mathbf{A}_d$ is invertible % TODO: I think we could also make it "if and only if"
    \item An unreachable DT system with non-invertible $\mathbf{A}_d$ could be controllable (Eigenvalues at 0). E.g.:
          \begin{itemize}
              \item $\mathbf{x}[k+1] = 0\mathbf{x}[k] + 0\mathbf{u}[k]$ is controllable (state goes to $0$) but unreachable as $\det(\mathbf{R})=0$.
          \end{itemize}
\end{itemize}

\subsection{Observability}

An LTI system is observable if, for any initial condition $\mathbf{x}_0$, this initial condition can be reconstructed uniquely based on the knowledge of the input and output signal over a finite time interval. This means that each state must influence the output.
\noindent\begin{align*}
    \mathbf{O} := & \begin{bmatrix}
                        \mathbf{C}    \\
                        \mathbf{CA}   \\
                        \mathbf{CA}^2 \\
                        \vdots        \\
                        \mathbf{CA}^{n-1}
                    \end{bmatrix}                                                                                   \\
    \mathbf{Y} =  & \; \mathbf{Ox}[n],\quad \mathbf{Y}={\left[\mathbf{y}[0],\ldots, \mathbf{y}[n]\right]}^{\mathsf{T}}
\end{align*}
Mathematically expressed, a system is observable if and only if the observability matrix $\mathbf{O}$ has \textbf{full column rank} $n$.
\begin{align*}
    \text{rank}(\mathbf{O}) & = n                     \\
    \mathbf{x}              & \notin \ker(\mathbf{O})
\end{align*}

\newpar{}
\ptitle{Remarks}
\begin{itemize}
    \item Note that $\mathbf{x}$ must not be in the null space of $\mathbf{O}$ so $\mathbf{Ox}=\mathbf{0}$ iff $\mathbf{x}=\mathbf{0}$ (trivial null space). % Sorry Tobi, kein nichttrivaler Nullraum à la H. Bölcskei ^^
\end{itemize}

\ptitle{SISO Diagonal Systems}

The $i$-th mode of a system in diagonal form is observable only if $\tilde{c}_i \neq 0$ because
\begin{equation*}
    y=\tilde{c}_1\tilde{x}_1+\ldots+\tilde{c}_n\tilde{x}_n+\mathbf{Du} % ChkTeX 11
\end{equation*}
This is a necessary but not sufficient condition for observability of the whole system.

\subsection{Modal View: Stabilizability and Detectability}
In a system that is stabilizable and detectable all the modes that cannot be controlled and/or observed must behave ``nicely'' on their own.

\newpar{}
\ptitle{Stabilizability}
\begin{itemize}
    \item A system is stabilizable if all \textbf{unstable modes are reachable}.
          \begin{itemize}
              \item I.e.\ it must be possible to bring all unstable modal components from $\mathbf{0}$ to any desired state in finite time.
          \end{itemize}
    \item Reachability always implies stabilizability
          \begin{itemize}
              \item A stabilizable system can be unreachable: E.g.\ if stable modes can't be influenced from the input.
          \end{itemize}
\end{itemize}

\newpar{}
\ptitle{Detectability}
\begin{itemize}
    \item A system is detectable if all \textbf{unstable modes are observable}.
          \begin{itemize}
              \item I.e.\ unstable modal behavior must be visible at the output
          \end{itemize}
    \item Observability always implies detectability
          \begin{itemize}
              \item A detectable system can be unobservable: E.g.\ if stable modes don't influence the output.
          \end{itemize}
\end{itemize}

\subsection{Kalman Decomposition}
\begin{center}
    \includegraphics[width = 0.7\linewidth]{Kalman_decomp.png}
\end{center}
A specific method to group basis vectors.

\newpar{}
\ptitle{Diagonalizable system}
\begin{align*}
    \dot{\mathbf{x}} & = \begin{bmatrix}
                             \bm{\Lambda}_{r\bar{o}} & 0                 & 0                             & 0                       \\
                             0                       & \bm{\Lambda}_{ro} & 0                             & 0                       \\
                             0                       & 0                 & \bm{\Lambda}_{\bar{r}\bar{o}} & 0                       \\
                             0                       & 0                 & 0                             & \bm{\Lambda}_{\bar{r}o} \\
                         \end{bmatrix}
    \mathbf{x} + \begin{bmatrix}
                     \mathbf{B}_{r\bar{o}} \\
                     \mathbf{B}_{ro}       \\
                     0                     \\
                     0                     \\
                 \end{bmatrix}
    \mathbf{u}                                                                                                                 \\
    \mathbf{y}       & = \begin{bmatrix}
                             0 & \mathbf{C}_{ro} & 0 & \mathbf{C}_{r\bar{o}}
                         \end{bmatrix}
    + \mathbf{Du}
\end{align*}

\ptitle{General case}
\begin{align*}
    \dot{\mathbf{x}} & = \begin{bmatrix}
                             \mathbf{A}_{r\bar{o}} & \mathbf{A}_{12} & \mathbf{A}_{13}             & \mathbf{A}_{14}       \\
                             0                     & \mathbf{A}_{ro} & 0                           & \mathbf{A}_{24}       \\
                             0                     & 0               & \mathbf{A}_{\bar{r}\bar{o}} & \mathbf{A}_{34}       \\
                             0                     & 0               & 0                           & \mathbf{A}_{\bar{r}o} \\
                         \end{bmatrix}
    \mathbf{x} + \begin{bmatrix}
                     \mathbf{B}_{r\bar{o}} \\
                     \mathbf{B}_{ro}       \\
                     0                     \\
                     0                     \\
                 \end{bmatrix}
    \mathbf{u}                                                                                                         \\
    \mathbf{y}       & = \begin{bmatrix}
                             0 & \mathbf{C}_{ro} & 0 & \mathbf{C}_{r\bar{o}}
                         \end{bmatrix}
    + \mathbf{Du}
\end{align*}

\newpar{}

\begin{tabularx}{\linewidth}{@{}ll@{}}
    $r$       & reachable      \\
    $\bar{r}$ & not reachable  \\
    $o$       & observable     \\
    $\bar{o}$ & not observable \\
\end{tabularx}

\ptitle{Remarks}
\begin{itemize}
    \item In the transfer function $u \rightarrow y$ only the modes corresponding to the reachable and observable modes will appear (others will be cancelled by a zero).
    \item A minimal realization of a transfer function is a state-space model that is both reachable and observable.
\end{itemize}