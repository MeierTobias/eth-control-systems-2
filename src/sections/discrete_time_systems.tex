\section{Discrete-Time Systems}
\subsection{Digital Control Systems}
All continuous-time (CT) system properties hold for discrete-time (DT) in the same way as DT values form a subset of CT values.
\subsubsection{Holding, D \textrightarrow{} A}
\textbf{Hold}: Linear system that takes DT signal as input and outputs CT signal.

Holding results in a CT signal with average delay $T/2$.
\newpar{}

\ptitle{Zero-Order Hold (ZOH)}
\begin{align*}
    y(t) & =u[k],                                                &  & \forall \; kT \le t < (k+1)T, \;k \in \mathbb{Z} \\
    y(t) & =u\left[\left\lfloor \frac{t}{T}\right\rfloor\right], &  & \forall t \in \mathbb{R}
\end{align*}

\subsubsection{Sampling, A \textrightarrow{} D}
\textbf{Sampler}: A linear system that takes CT signal as input and outputs DT sequence.
\begin{equation*}
    y[k]=u(kT),\quad\forall k\in\mathbb{Z}
\end{equation*}

\subsubsection{Aliasing}
\ptitle{Nyquist-Shannon Sampling Theorem}

A signal can be reconstructed without aliasing from samples at rate $1/T$ if it does not contain frequencies higher than the Nyquist (folding) frequency:
\begin{equation*}
    \omega_{\mathbf{f}}:=\frac\pi T[rad/s]=\frac1{2T}[Hz]
\end{equation*}
A signal
\begin{equation*}
    y[k]=\cos[\omega\cdot kT]
\end{equation*}
has the same samples as a sinusoid with frequency
\begin{equation*}
    \omega'=\left|\omega\pm n\frac{2\pi}{T}\right|,\; n\in\mathbb{N}
\end{equation*}

\ptitle{Anti-Aliasing Filter}

Low-pass filter before the A $\rightarrow$ D converter with
\begin{equation*}
    \omega_{c,LP} < \omega_f
\end{equation*}
where $\omega_{c,LP}$ must be high enough to capture system dynamics.

\subsection{DT State Space Models}
\subsubsection{Difference Equations}

\renewcommand{\arraystretch}{1.3}
\setlength{\oldtabcolsep}{\tabcolsep}\setlength\tabcolsep{3pt}

\begin{tabularx}{\linewidth}{@{}lll@{}}
                                                             & Time-Invariant                                     & Time-Variant ($t = kT$)                                                   \\
    \cmidrule{2-3}
    \multirow{3}{*}{\begin{sideways}Nonlinear\end{sideways}} & $\mathbf{x}[k+1] = f(\mathbf{x}[k],\mathbf{u}[k])$ & $\mathbf{x}[k+1] = f(t,\mathbf{x}[k],\mathbf{u}[k])$                      \\
                                                             & $\mathbf{y}[t] = h(\mathbf{x}[k],\mathbf{u}[k])$   & $\mathbf{y}[t] = h(t,\mathbf{x}[k],\mathbf{u}[k])$                        \\
                                                             & $\mathbf{x}[0]=\mathbf{x}_0$                       & $\mathbf{x}[0]=\mathbf{x}_0$                                              \\
    \cmidrule{2-3}
    \multirow{3}{*}{\begin{sideways}Linear\end{sideways}}    & $\mathbf{x}[k+1] = \mathbf{Ax}[k]+\mathbf{Bu}[k]$  & $\mathbf{x}[k+1] = \mathbf{A}[k]\mathbf{x}[k]+\mathbf{B}[k]\mathbf{u}[k]$ \\
                                                             & $\mathbf{y}[k] = \mathbf{Cx}[k]+\mathbf{Du}[k]$    & $\mathbf{y}[k] = \mathbf{C}[k]\mathbf{x}[k]+\mathbf{D}[k]\mathbf{u}[k]$   \\
                                                             & $\mathbf{x}[0] = \mathbf{x}_0$                     & $\mathbf{x}[0]=\mathbf{x}_0$
\end{tabularx}

\renewcommand{\arraystretch}{1}
\setlength{\tabcolsep}{\oldtabcolsep}

As in CT, the system is \textbf{strictly causal} iff $\mathbf{D}=0$.

\subsubsection{Time Response}
\noindent\begin{equation*}
    y[k]=\underbrace{\mathbf{CA}^{k}\mathbf{x}_0}_{\textsf{homogeneous r.}} + \underbrace{\mathbf{C}\sum_{i=0}^{k-1}\mathbf{A}^{k-i-1}\mathbf{Bu}[i]+\mathbf{Du}[k]}_{\textsf{forced response}}
\end{equation*}


\subsubsection{Internal Stability for DT Systems}
\ptitle{If A is Diagonalizable}

For diagonalizable matrices A:\
\begin{equation*}
    \mathbf{A}^{k} ={(\mathbf{T}\bm{\Lambda}\mathbf{T}^{-1})}^k=\mathbf{T}\bm{\Lambda}^{k}\mathbf{T}^{-1}=\mathbf{T}
    \begin{bmatrix}
        \lambda_1^k &             &        &             \\
                    & \lambda_2^k &        &             \\
                    &             & \ddots &             \\
                    &             &        & \lambda_n^k
    \end{bmatrix}\mathbf{T}^{-1}
\end{equation*}
$\lim_{k\to+\infty}\mathbf{A}^k=0$ is fulfilled if
\begin{equation*}
    |\lambda_i| <1,\quad\forall i=1,\ldots,n
\end{equation*}
which means that all eigenvalues must lie within the unit circle for the system to be stable.

Reminder:
\begin{equation*}
    \exp\left(
    \begin{bmatrix}
        \lambda_1 & 0         \\
        0         & \lambda_2
    \end{bmatrix}
    t\right) =
    \begin{bmatrix}
        e^{\lambda_{1}t} & 0                \\
        0                & e^{\lambda_{2}t}
    \end{bmatrix}
\end{equation*}

\ptitle{If A is Defective}

Similarly, each block in the Jordan form will converge to zero if and only if
\begin{equation*}
    |\lambda_i| <1,\quad\forall i=1,\ldots,n
\end{equation*}
Reminder:
\begin{align*}
    \left.\exp\left(
    \begin{bmatrix}\lambda & 1       \\
               0       & \lambda \\
        \end{bmatrix}\right.t\right)           & =
    \begin{bmatrix}1 & t \\
               0 & 1
    \end{bmatrix}
    \text{e}^{\lambda t}                       \\
    \left.\exp\left(
    \begin{bmatrix}\lambda & 1       & 0       \\
               0       & \lambda & 1       \\
               0       & 0       & \lambda
        \end{bmatrix}\right.t\right) & =
    \begin{bmatrix}1 & t & \frac{1}{2!}t^2 \\
               0 & 1 & t               \\
               0 & 0 & 1
    \end{bmatrix}
    \text{e}^{\lambda t}
\end{align*}

\subsection{Transfer Functions (TF) of DT LTI Systems}\label{disc:tf}
\ptitle{Time Response to Elementary Inputs}

Similar to the elementary inputs in the CT case, a \textit{geometric sequence} $z^k$ corresponds to an input sequence
\begin{equation*}
    u[k] =u_0z^k=u_0e^{ksT}=u(kT), \; z=e^{sT}
\end{equation*}
The corresponding time response is
\begin{align*}
    y[k] & = \underbrace{\mathbf{CA}^k(\mathbf{x}_0-\mathbf{C}{(z\mathbf{I}-\mathbf{A})}^{-1}\mathbf{B}u_0)}_{\textsf{transient}}+\dots \\
         & +\underbrace{\mathbf{C}{(z\mathbf{I}-\mathbf{A})}^{-1}\mathbf{B}u_0z^k+\mathbf{D}u_0z^k}_{\textsf{steady-state}}
\end{align*}

\ptitle{Pulse/Discrete TF}

For asymptotically stable DT LTI systems, for large $k$ and $u[k]=u_0 z^k$:
\begin{align*}
    y[k] & \approx G(z)u[k]                                                 \\
    G(z) & :=\mathbf{C}{(z\mathbf{I}-\mathbf{A})}^{-1}\mathbf{B}+\mathbf{D}
\end{align*}

\newpar{}
\ptitle{Remarks}
\begin{itemize}
    \item $z^k$ can be interpreted as samples from an exponential
\end{itemize}

\subsubsection{Special TF and Realizations}
\ptitle{DC Gain}

If $G(z)$ is asymptotically stable and $\mathbf{A}$ is invertible, i.e.\ has no eigenvalue at 1, then
\begin{equation*}
    \lim_{k\to+\infty}y[k]=G(1)u_0=(\mathbf{C}{(\mathbf{I}-\mathbf{A})}^{-1}\mathbf{B}+\mathbf{D})u_0
\end{equation*}

\ptitle{Time Delay}

\begin{equation*}
    y[k] =u[k-1] =u_0z^{k-1}=\frac1zu[k]
\end{equation*}

\ptitle{Realizations of DT TF}

All methods to create (state space) realizations of CT TF as LTI models can be adapted to DT TF.\

\subsection{System Discretization}

Similar to CT, the natural dynamics of a DT LTI system are given by the eigenvalues of $\mathbf{A}_d$.

For given eigenvalues $\lambda_i$ of $\mathbf{A}$ the eigenvalues of $\mathbf{A}_d$ are
\begin{equation*}
    z_i=e^{\lambda_1T},e^{\lambda_2T},\ldots,e^{\lambda_{n}T}
\end{equation*}

\ptitle{Mapping}

\begin{enumerate}
    \item CT left half plane $\rightarrow$ DT unit disk
    \item CT neighborhood of origin $\rightarrow$ DT neighborhood of $+1$ point
    \item CT poles at $j\omega_f$ $\rightarrow$ DT $-1$ point (because $e^{j\omega_{f}T}=-1$)
    \item CT finite poles cannot be mapped to the DT origin and the DT origin does not have a CT equivalent
\end{enumerate}

\begin{center}
    \includegraphics[width=0.9\linewidth]{mapping.png}
\end{center}


\subsubsection{Discretization: Hold and Sample}\label{disc::hold_and_sample}

To discretize the $\mathbf{A}, \mathbf{B}, \mathbf{C}, \mathbf{D}$ matrices of a time continuous system one has to apply the following transformations:
\begin{align*}
    \mathbf{A}_d & :=e^{\mathbf{A}T} & \mathbf{B}_d & :=\left(\int_0^{\mathsf{T}} {e^{\mathbf{A}\theta}}d\theta\right)\mathbf{B} \overset{\mathbf{A} \text{ inv.}}{=} \mathbf{A}^{-1}\left(\mathbf{A}_d-\mathbf{I}\right)\mathbf{B} \\
    \mathbf{C}_d & :=\mathbf{C}      & \mathbf{D}_d & :=\mathbf{D}
\end{align*}
%Alternatively the DT state space matrices $\mathbf{A}_d, \mathbf{B}_d, \mathbf{C}_d, \mathbf{D}_d$ can be used to compute the DT transfer function $G_{d}(z)$ and the corresponding difference equation can be derived using the z transform. 
%TODO: This belongs to the section "Transfer Functions (TF) of DT LTI Systems" I would suggest we add the subscript _d to the formulas there (which probably caused the confusion) and remove this comment. We discussed it and Dani wil have a look at it and will decide if we delete it or move it

\newpar{}
\ptitle{Calculations}
\begin{itemize}
    \item Check invertibility by looking at the determinant.
    \item For diagonal matrices $\Lambda$, one can apply exponentiation elementwise on the main diagonal.
    \item For non-diagonal matrices use
          \begin{itemize}
              \item either diagonalization via eigenvectors ($A=V^{-1}\cdot \Lambda \cdot V$) and
                    \begin{equation*}
                        e^{AT} = V^{-1}\cdot e^{\Lambda T} \cdot V
                    \end{equation*}
              \item or power series expansion
                    \begin{equation*}
                        e^{AT}=\sum_{k=0}^{\infty}\frac{1}{k!}{\left(AT\right)}^k
                    \end{equation*}
          \end{itemize}
\end{itemize}

\subsubsection{Discretization: Frequency Domain}\label{disc::tustin}
\ptitle{Euler Forward}
\noindent\begin{equation*}
    s = \frac{z-1}{T}
\end{equation*}
\ptitle{Euler Backward}
\noindent\begin{equation*}
    s = \frac{z-1}{zT}
\end{equation*}

\ptitle{Tustin's Method}
\begin{equation*}
    G_{d}(z)=G\left(s=\frac2T\frac{z-1}{z+1}\right)
\end{equation*}
which corresponds to a rearranged Padé approximation for $z$.

\subsubsection{Discretization: Time Domain}

\paragraph{Forward Difference}
\noindent\begin{align*}
    \mathbf{x}(t)-\mathbf{x}(t-T) & \approx T\cdot\dot{\mathbf{x}}(t-T) =T\cdot(\mathbf{Ax}(t-T)+\mathbf{Bu}(t-T)) \\
                                  & \text{yields}                                                                  \\
    \mathbf{x}(t)                 & \approx(\mathbf{I}+\mathbf{A}T)\mathbf{x}(t-T)+T\mathbf{Bu}(t-T)
\end{align*}

\begin{tabularx}{\linewidth}{@{}m{0.6\linewidth}X@{}}
    which is simple but can result in an unstable compensator as
    \begin{equation*}
        s=\lambda \rightarrow z=1+\lambda T
    \end{equation*}
    i.e.\ for large $|\lambda|$ there is no mapping into the unit disk.
     &
    \includegraphics[width=0.9\linewidth, align=m]{disc_approx_forward.pdf}
\end{tabularx}

\paragraph{Backward Difference}
\noindent\begin{align*}
    \mathbf{x}(t)-\mathbf{x}(t-T) & \approx T\cdot\dot{\mathbf{x}}(T) =T\cdot(\mathbf{Ax}(t)+\mathbf{Bu}(t))                             \\
                                  & \text{yields}                                                                                        \\
    \mathbf{x}(t)                 & \approx{(\mathbf{I}-\mathbf{A}T)}^{-1}\mathbf{x}(t-T)+{(\mathbf{I}-\mathbf{A}T)}^{-1}T\mathbf{Bu}(t)
\end{align*}

\begin{tabularx}{\linewidth}{@{}m{0.6\linewidth}X@{}}
    which avoids instability as
    \begin{equation*}
        s=\lambda \rightarrow z={(1-\lambda T)}^{-1}
    \end{equation*}
    This transforms poles into a disk smaller than the unit disk. DT poles cannot get negative real part.
     &
    \includegraphics[width=0.9\linewidth, align=m]{disc_approx_backwards.pdf}
\end{tabularx}

\paragraph{Trapezoidal Rule}\label{disc::trapezoid}
\noindent\begin{align*}
    \mathbf{x}(t)-\mathbf{x}(t-T) & \approx \frac T2\left(\dot{\mathbf{x}}(t)+\dot{\mathbf{x}}(t-T)\right)                                                  \\
                                  & \text{yields}                                                                                                           \\
    \mathbf{x}(t)                 & \approx{\left(\mathbf{I}-\frac{\mathbf{A}T}{2}\right)}^{-1}\left(\mathbf{I}+\frac{\mathbf{A}T}{2}\right)\mathbf{x}(t-T) \\
                                  & +{\left(\mathbf{I}-\frac{\mathbf{A}T}{2}\right)}^{-1}\mathbf{B}T\frac{u(t)+u(t-T)}2
\end{align*}

\begin{tabularx}{\linewidth}{@{}m{0.6\linewidth}X@{}}
    which maps the left half plane to the unit disk. This discretization method corresponds to Tustin's method in frequency domain.
     &
    \includegraphics[width=0.9\linewidth, align=m]{disc_approx_trapezoid.pdf}
\end{tabularx}

\subsection{Design and Implementation of DT Control Systems}
There are two main design paradigms for DT control systems:
\begin{enumerate}
    \item DT:\ The CT system is converted to a DT system and the design process happens in DT
    \item Emulation:\ the design process happens in CT and the result is transferred to DT
\end{enumerate}

\begin{center}
    \includegraphics[width=0.9\linewidth]{digit_controller_design.pdf}
\end{center}

\newpar{}
\ptitle{Remarks}

\begin{itemize}
    \item A common method is to
          \begin{itemize}
              \item Use DT design if $T$ is known
              \item Use CT design with emulation if $T$ is unknown
          \end{itemize}
\end{itemize}

\subsubsection{DT}
\begin{enumerate}
    \item First the CT model of the plant is transformed into a DT model (use the ``Hold and Sample'' discretization method as shown in~\ref{disc::hold_and_sample}).
    \item Then the discrete compensator can be designed.
\end{enumerate}

\subsubsection{Emulation}
\begin{enumerate}
    \item First the compensator is designed in the CT domain.
    \item This compensator is then transformed into a discrete compensator with one of the following methods:
          \begin{itemize}
              \item Frequency domain \textrightarrow{} Tustin's method (\ref{disc::tustin})
              \item Time domain \textrightarrow{} Trapezoid method (\ref{disc::trapezoid})
          \end{itemize}
\end{enumerate}
