\section{Nonlinear Systems}

\subsection{Concepts of Stability}
A nonlinear system
\noindent\begin{align*}
    \dot{\mathbf{x}}(t) & = f(\mathbf{x}(t), \mathbf{u}(t)) & \mathbf{x}(0)=\mathbf{x}_0 \\
    \mathbf{y}(t)       & = h(\mathbf{x}(t), \mathbf{u}(t))
\end{align*}
\textbf{does not satisfy the superposition property} and therefore
\begin{itemize}
    \item Effects from inputs and from initial conditions caannot be seperated.
    \item The input cannot be seperated into elementary inputs and the output will not be a composition of elementary outputs.
\end{itemize}

\newpar{}
\ptitle{Properties}
\begin{itemize}
    \item A nonlinear system can have zero, one or multiple \textbf{equillibrium points} s.t.\ $f(\mathbf{x}_e, 0)=0$.
    \item In contrast to linear systems that require infinite time to go to infinity, nonlinear systes can \textbf{go to infinity in finite time}.
    \item Nonlniear systems may generate permanent oscillations of fixed amplitude - \textbf{limit cycles} - independent of initial conditions.
    \item Nonlinear systems can generate outputs at frequencies that are submultiples or multiples of the input frequency (\textbf{Subharmonic oscillations}).
    \item Deterministic nonlinear systems can generate \textbf{chaos}.
\end{itemize}

\subsection{Lyapunov Stability Theory}
\subsubsection{Types of Stability}
Assuming $\mathbf{u}=0, \mathbf{x}_e=0$, the eqilibrium $\mathbf{x}_e$ of the systems
\noindent\begin{equation*}
    \dot{\mathbf{x}} = f(\mathbf{x}), \quad f(\mathbf{x}_e) = 0
\end{equation*}
is
\begin{itemize}
    \item \textbf{Stable in the sense of Lyapunov}
          \noindent\begin{equation*}
              \|\mathbf{x}(0)\| < \delta \Rightarrow \|\mathbf{x}(t)\| \leq \varepsilon,\qquad \forall t\geq 0,\; \delta > 0,\; \varepsilon\geq0
          \end{equation*}
    \item \textbf{Asymtotically stable}
          \noindent\begin{equation*}
              \|\mathbf{x}(0)\| < \delta \Rightarrow \lim_{t\to +\infty} \mathbf{x}(t)=0, \qquad \delta>0
          \end{equation*}
    \item \textbf{Exponentially stable}
          \noindent\begin{equation*}
              \|\mathbf{x}(0)\| < \delta \Rightarrow \|\mathbf{x}(t)\| < \beta e^{-\alpha t}, \qquad \forall t\geq 0,\; \alpha, \beta, \delta >0
          \end{equation*}
\end{itemize}

\subsubsection{Lyapunov Functions}
\textit{Lyapunov functions are, in a sense, a notion of energy: non-negative, minimized (0) only at the quilibrium point and non-increasing along all trajectories.}
\newpar{}
For a given compact subset $D$ of the state space containing $\mathbf{x}_e$, a Lypunov function is defined as
\noindent\begin{equation*}
    V:D \to \mathbb{R} \mapsto V(\mathbf{x})
\end{equation*}

If this Lyapunov function satisfies
\noindent\begin{align*}
    V(\mathbf{x})                 & \geq 0                                                                                                                                                          &  & \forall \mathbf{x}\in D    \\
    V(\mathbf{x})                 & = 0 \Leftrightarrow \mathbf{x}_e = \mathbf{x}                                                                                                                                                   \\
    \frac{d}{dt} V(\mathbf{x}(t)) & = \frac{\partial V(\mathbf{x})}{\partial t} \frac{\partial \mathbf{x}(t)}{\partial \mathbf{x}} = \frac{\partial V(\mathbf{x})}{\partial t} f(\mathbf{x}) \leq 0 &  & \forall \mathbf{x}(t)\in D
\end{align*}
the equillibrium point $\mathbf{x_e}$ is \textbf{stable in the sense of Lyapunov}.

\newpar{}
Futhermore, $\mathbf{x}_e$ is \textbf{asymptotically stable} if
\noindent\begin{equation*}
    \dot{V}(\mathbf{x}(t)) = 0 \Leftarrow \mathbf{x}(t) = \mathbf{x}_e
\end{equation*}
and \textbf{exponentially stable} if
\noindent\begin{equation*}
    \dot{V}(\mathbf{x}(t)) < -\alpha V(\mathbf{x}(t)), \qquad \alpha>0
\end{equation*}

\paragraph{Global Stability}
To ensure global stability, i.e. $D=\mathbb{R}^n$, a Lyapunov function $V$ satisfying the aforementioned properties needs to be \textbf{radially unbounded}:
\noindent\begin{equation*}
    \|\mathbf{x}\| \to +\infty \Rightarrow V(\mathbf{x}) \to +\infty
\end{equation*}

\paragraph{Indirect Method}
According to the \textbf{Hartman-Grobman Theorem}, a linearized system
\noindent\begin{equation*}
    A = \left.\frac{\partial f(\mathbf{x})}{\partial \mathbf{x}} \right|_{\mathbf{x} = 0}
\end{equation*} with no unstable pole behaves qualitatively the same as the nonlinear around the equilibrium point.

\newpar{}
As a result, the Lyapunov function of the linear system also describes the nonlinear system for sufficiently small deviations from the equilibrium point.

\newpar{}
\textbf{Remark}

The solution $\mathbf{P}$ of the Riccati equation can be interpreted as a Lyapunov equation:
\noindent\begin{equation*}
    V(\mathbf{x}) = \mathbf{x}^{\mathsf{T}} \mathbf{Px}
\end{equation*}

\paragraph{LaSalle's Invariance Principle}
LaSalle's invariance theorem is useful if Lyapunov stability is given by the existence of a Lyapunov function but $\dot{V}=0$ not only at $\mathbf{x}_e$:

\newpar{}
\textit{Let $\Omega\in D$ be a (compact) positively invariant set i.e., $\mathbf{x}(t_0)\in \Omega \Rightarrow \mathbf{x}(t)\in\Omega$.
    Let $V_D\to \mathbb{R}^n$, such that $\dot{V}(\mathbf{x})\leq 0\; \forall \mathbf{x}\in \Omega$ (i.e.\ there could be multiple points with $\dot{V}=0$).
    Then, $\mathbf{x}(t)$ will eventually approach the largest positively invariant set in which $\dot{V}=0$
}
\newpar{}
\textbf{Remarks}
\begin{itemize}
    \item A set $\Omega$ is positively invariant if, once the system's state $\mathbf{x}(t)$ enters this set, it will stay in this set for all future time $t \geq t_0$. This means if $\mathbf{x}(t_0) \in \Omega$, then $\mathbf{x}(t) \in \Omega\; \forall t \geq t_0$.
    \item For the pendulum, one has multiple states (turning points, origin) where $\dot{\mathbf{x}}=0$ but $\mathbf{x}(t)$ only stays at the origin (but not at the turning points) once it has ``entered'' this equilibrium set (i.e.\ stopped moving).
\end{itemize}

\subsection{Control Lyapunov Functions}
If a Control Lyapunov Function (CLF) exists, it provides a way to construct a stabilizing feedback with an appropriate control input $\tilde{\mathbf{u}}$ (\textit{Artstein's Theorem}).

\newpar{}
A CLF satisfies
\noindent\begin{align*}
    V(\mathbf{x})                                              & \geq 0                                                                                           &  & \text{positive definite}  \\
    V(\mathbf{x})                                              & = 0 \Leftrightarrow \mathbf{x} = 0                                                               &  & \text{radially unbounded} \\
    \frac{d}{dt} V(\mathbf{x}, \tilde{\mathbf{u}}(\mathbf{x})) & = \frac{\partial V(\mathbf{x})}{\partial t} f(\mathbf{x}, \tilde{\mathbf{u}}(\mathbf{x})) \leq 0 &  & \forall \mathbf{x}\neq 0
\end{align*}

\subsubsection{Gain Scheduling}
In gain scheduling, the state space is partitioned into disjoint subspaces and for a number of design equilibria ($\mathbf{x}_i, \mathbf{u}_i$)  nominal contorl laws $\mathbf{K}_i$ are designed using the aforementioned techniques.

\newpar{}
The dynamics of the system around a equilibrium point $\mathbf{x}_i$ can be approximated by
\noindent\begin{equation*}
    \dot{\mathbf{x}} = \mathbf{A}_i(\mathbf{x}-\mathbf{x}_i) + \mathbf{B}_i(\mathbf{u}-\mathbf{u}_i)
\end{equation*}

The stability of
\noindent\begin{equation*}
    \mathbf{u} = \mathbf{u}_i+\mathbf{K}_i(\mathbf{x}-\mathbf{x}_i)
\end{equation*}
can be proven with the Lyapunov function ($\forall i$)
\noindent\begin{equation*}
    V(\mathbf{x}) = \min_{i} {(\mathbf{x}-\mathbf{x}_i)}^{\mathsf{T}}\mathbf{P}(\mathbf{x}-\mathbf{x}_i)
\end{equation*}

\subsubsection{Multiple Lyapunov Functions}
Another option is to consider multiple Lyapunov functions $V_i$ with corresponding sets $X_i$ that satisfy
\begin{itemize}
    \item positive definite in $X_i$
    \item $\dot{V_i}(\mathbf{x})\leq 0$ when $\mathbf{x}\in X_i$
\end{itemize}

and to check if the system satisfies the \textbf{sequence non-increasing condition}:
\noindent\begin{equation*}
    V_i[k+1] < V_i[k] \qquad \forall k \in \mathbb{N}
\end{equation*}

\subsubsection{Linear Parameter-Varying Systems}
A third option is to rewrite the nonlinear system in a ``linear'' fashion with slow varying parameter $\sigma(\mathbf{x})$
\noindent\begin{equation*}
    \dot{\mathbf{x}}_\delta = \mathbf{A} (\sigma(\mathbf{x}))\mathbf{x}_\delta + \mathbf{B}(\sigma(\mathbf{x}))\mathbf{u}_\delta\quad
    \begin{cases}
        \mathbf{x}_\delta = \mathbf{x}-\mathbf{x}_e(\sigma(\mathbf{x})) \\
        \mathbf{u}_\delta = \mathbf{u}-\mathbf{u}_e(\sigma(\mathbf{x}))
    \end{cases}
\end{equation*}

This system is \textbf{exponentially stable} if
\noindent\begin{equation*}
    {\bar{\mathbf{A}}(\sigma)}^{\mathsf{T}} \mathbf{P} + \mathbf{P}\bar{\mathbf{A}}(\sigma) + \sum \rho_i \frac{\partial \mathbf{P}(\sigma)}{\partial \sigma_i} < 0,\qquad
    \begin{cases}
        \sigma(t)\in S       \\
        \dot{\sigma(t)}\in R \\
        \forall \sigma\in S, \rho\in R
    \end{cases}
\end{equation*}
In other words, the existence of a family of positive matrices $\mathbf{P}(\sigma)$ (common Lyapunov functions) prove the systems exponential stability.

\subsubsection{Backstepping Control}
Backstepping is a form of \textit{cascaded control} where a control input $\mathbf{u}$ is chosen s.t.\ the output $\mathbf{z}$ of the outer loop stabilizes the inner loop:

\noindent\begin{align*}
    \dot{\mathbf{x}} & = f_0(\mathbf{x}) + g_0(\mathbf{x})\mathbf{z}                         &  & \text{inner loop} \\
    \dot{\mathbf{z}} & = f_1(\mathbf{x},\mathbf{z}) + g_1(\mathbf{x}, \mathbf{z}) \mathbf{u} &  & \text{outer loop}
\end{align*}
where the inner system has an equilibrium point in $\mathbf{x}=0, \mathbf{z}=0$.

\newpar{}
If $\mathbf{z} = \mathbf{u}_0(\mathbf{x})$ then 
\noindent\begin{equation*}
    \frac{d}{dt} V_0(\mathbf{x}) = -W(\mathbf{x}) < 0
\end{equation*}
proving stability in the sense of Lyapounov.

\newpar{}
\ptitle{Controlling the Error}

In order to ensure stability, Lyapunov stability of the error $\mathbf{e} = \mathbf{z}-\mathbf{u}_0$ needs to be established.

The Lypunov candidate
\noindent\begin{equation*}
    V_1(\mathbf{x}, \mathbf{e}) = V_0(\mathbf{x})+\frac{1}{2} \mathbf{e}^2
\end{equation*}
satisfies
\noindent\begin{equation*}
    \frac{d}{dt}V_1(\mathbf{x}, \mathbf{e}) = -W(\mathbf{x}) - k_1 \mathbf{e}^2 <0 ,\qquad \forall(\mathbf{x},\mathbf{e}) \neq 0
\end{equation*}
and thus proves the stability of the ``error'' system
\noindent\begin{align*}
    \dot{\mathbf{x}} &= f_0(\mathbf{x})+g_0(\mathbf{x})\mathbf{u}_0(\mathbf{x}) + g_0(\mathbf{x})\mathbf{e}\\
    \dot{\mathbf{e}} &= \dot{\mathbf{z}} - \underbrace{\frac{\partial \mathbf{u}_0(\mathbf{x})}{\partial \mathbf{x}}(\dot{\mathbf{x}})}_{\dot{\mathbf{u}}_0} = \mathbf{v}
\end{align*}