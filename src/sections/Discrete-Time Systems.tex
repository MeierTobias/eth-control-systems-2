\section{Discrete-Time Systems}
\subsection{Digital Control Systems}
All continuous-time (CT) system properties hold for discrete-time (DT) in the same way as DT values form a subset of CT values.
\subsubsection{Sub-Sub-Section}
%%
\ptitle{D $\rightarrow$ A: Zero-Order Hold (ZOH)}
\noindent\begin{align*}
    y(t) & =u[k], \; \forall kT \le t < (k+1)T                 \\
    y(t) & =u[\lfloor t/T\rfloor], \; \forall t \in \mathbb{R}
\end{align*}
Hold: a linear system that takes DT signal as input and outputs CT signal.\\
Holding results in a CT signal with average delay $T/2$.
%%
\noindent\begin{align*}
    y[k]= & u(kT),\quad\forall k\in\mathbb{Z}
\end{align*}
Sampler: a linear system that takes CT signal as input and outputs DT sequence.\\
%%
\subsubsection{Aliasing}
\ptitle{Nyquist-Shannon Sampling Theorem}

A signal can be reconstructed without aliasing from samples at rate $1/T$ if it does not contain frequencies higher than the Nyquist (folding) frequency:
\noindent\begin{align*}
    \omega_{\mathbf{f}}: & =\frac\pi T[rad/s]=\frac1{2T}[Hz]
\end{align*}
A signal
\noindent\begin{align*}
    y[k] & =\cos[\omega\cdot kT]
\end{align*}
has the same samples as a sinusoid with frequency
\noindent\begin{align*}
    \omega' & =\left|\omega\pm n\frac{2\pi}{T}\right|,\; n\in\mathbb{N}
\end{align*}
%
\ptitle{Anti-Aliasing Filter}

Low-pass filter before the A $\rightarrow$ D converter with
\noindent\begin{align*}
    \omega_{c,LP} & < \omega_f
\end{align*}
where $\omega_{c,LP}$ must be high enough to caputre system dynamics.
%
%
\subsection{DT State-Space Models}
\subsubsection{Difference Equations}
\ptitle{General Case}

A finite-dimensional, DT, causal system can be modelled by
\noindent\begin{align*}
    x[k+1] & =f(t,x[k],u[k]), \; x[0]=x_0 \\
    y[t]   & =h(t,x[k],u[k])
\end{align*}
If the system is additionally linear it can be modelled by
\noindent\begin{align*}
    x[k+1] & =A[k]x[k]+B[k]u[k], \; x[0]=x_0 \\
    y[k]   & =C[k]x[k]+D[k]u[k]
\end{align*}
A LTI system can be modelled by
\noindent\begin{align*}
    x[k+1] & =Ax[k]+Bu[k], \; x[0]=x_0 \\
    y[k]   & =Cx[k]+Du[k]
\end{align*}
As in CT, the system is \textbf{strictly causal} iff $D=0$.
\subsubsection{Time Response}
\noindent\begin{align*}
    y[k]=\underbrace{CA^kx_0}_{\text{homogeneous r.}} + \underbrace{C\sum_{i=0}^{k-1}A^{k-i-1}Bu[i]+Du[k]}_{\text{forced response}}
\end{align*}
\subsubsection{Internal Stability for DT Systems}
\ptitle{If A is Diagonalizable}

For diagonalizable matrices A:
\noindent\begin{align*}
    A^k & =(T\Lambda T^{-1})^k=T\Lambda^kT^{-1}=T\begin{bmatrix}\lambda_1^k&&&\\&\lambda_2^k&&\\&&\ddots&\\&&&\lambda_n^k\end{bmatrix}T^{-1}
\end{align*}
$\lim_{k\to+\infty}A^k=0$ is fulfilled if
\noindent\begin{align*}
    |\lambda_i| & <1,\quad\forall i=1,\ldots,n
\end{align*}
which means that all Eigenvalues must lie within the unit circle in the complex plane.\\
\ptitle{If A is Defective}

Similarly, each block in the Jordan form will converge to zero if and only if
\noindent\begin{align*}
    |\lambda_i| & <1,\quad\forall i=1,\ldots,n
\end{align*}
Reminder:
\noindent\begin{align*}
    \left.\exp\left(\begin{bmatrix}\lambda&1\\ 0&\lambda\\\end{bmatrix}\right.t\right)               & =\begin{bmatrix}1&t\\0&1\end{bmatrix}\text{e}^{\lambda t}                          \\
    \left.\exp\left(\begin{bmatrix}\lambda&1&0\\0&\lambda&1\\0&0&\lambda\end{bmatrix}\right.t\right) & =\begin{bmatrix}1&t&\frac{1}{2!}t^2\\0&1&t\\0&0&1\end{bmatrix}\text{e}^{\lambda t}
\end{align*}
%
\subsection{Transfer Functions (TF) of DT LTI Systems}
\ptitle{Time Response to Elementry Inputs}

Similar to the CT case, a geometric sequence $z^k$ corresponds to an input sequence
\noindent\begin{align*}
    u[k] & =u_0z^k=u_0e^{ksT}=u(kT), \; z=e^{sT}
\end{align*}
The corresponding time response is
\noindent\begin{align*}
    y[k] & = \underbrace{CA^k(x_0-C(zI-A)^{-1}Bu_0)}_{\mathsf{transient}}+ \\\underbrace{C(zI-A)^{-1}Bu_0z^k+Du_0z^k}_{\mathsf{steady-state}}
\end{align*}
\ptitle{Pulse/Discrete TF}

For asymptotically stable DT LTI systems, for large $k$:
\noindent\begin{align*}
    y[k] & \approx G(z)u[k]  \\
    G(z) & :=C(z1-A)^{-1}B+D
\end{align*}
\subsubsection{Special TF and Remarks}
\ptitle{DC Gain}

If it exists (no eigenvalue at $1$), the real matrix
\noindent\begin{align*}
    G(1) & =C(1-A)^{-1}B+D
\end{align*}
is called DC gain and corresponds to $\lim_{k\to+\infty}y[k]$ for $z=1$ i.e. $u[k]=u_0$.\\
%
\ptitle{Time Delay}

\noindent\begin{align*}
    y[k] & =u[k-1] & =u_0z^{k-1}=\frac1zu[k]
\end{align*}
\ptitle{Realizations of DT TF}

All methods to create (state space) realizations of CT TF as LTI models can be adapted to DT TF.
%
%
\newpage
\subsection{Design and Implementation of DT Control Systems}
\subsubsection{Basics}
There are two main design paradigms for DT control systems:
\begin{enumerate}
    \item DT: The CT system is converted to a DT system and the design process happens in DT
    \item Emulation: the design process happens in CT and the result is transferred to DT
\end{enumerate}
\ptitle{Derivation of DT State Space Matrices}

\noindent\begin{align*}
    A_d & :=e^{AT}                                                   \\
    B_d & :=\left(\int_0^T {e^{A\theta}}d\theta\right)B              \\
        & = A^{-1}\left(A_d-I\right)B\;\; \text{(if $A$ invertible)} \\
    C_d & :=C                                                        \\
    D_d & =D                                                         \\
\end{align*}
\paragraph{s Plane vs. z Plane}
Similar to CT, the natural dynamics of a CT LTI system are given by the Eigenvalues of $A_d$.\\
For given Eigenvalues $\lambda_i$ of $A$ the eigenvalues of $A_d$ are
\noindent\begin{align*}
    z & =e^{\lambda_1T},e^{\lambda_2T},\ldots,e^{\lambda_nT}
\end{align*}
\ptitle{Mapping}

\begin{enumerate}
    \item CT left half plane $\rightarrow$ DT unit disk
    \item CT neighborhood of origin $\rightarrow$ DT neighborhood of $+1$ point
    \item CT poles at $j\omega_f$ $\rightarrow$ DT $-1$ point (because $e^{j\omega_fT}=-1$)
    \item CT finite poles cannot be mapped to the DT origin and the DT origin does not have a CT equivalent
\end{enumerate}
\subsubsection{Emulation}
Design the control system in CT. This outputs a compensator (as TF or state space model) which is then converted to DT.
\paragraph{Emulation in Frequency Domain}
\ptitle{Tustin's Method}

\noindent\begin{align*}
    G_{c,d}(z) & =G_{c}\left(s=\frac2T\frac{z-1}{z+1}\right)
\end{align*}
which corresponds to a rearranged Padé approximation for $z$.
\paragraph{Emulation in Time Domain}
If the design process outputs a state space controller the resulting matrices could easily be transferred to DT.\\
However, the "hold and sample" does not work for the compensator as its inputs are DT and the outputs are held (not vice versa).\\
\ptitle{Numerical Emulation: Forward Difference}

\noindent\begin{align*}
    \xi(t)-\xi(t-T) & \approx T\cdot\dot{\xi}(t-T)   \\
                    & \text{yields}                  \\
    \xi(t)          & \approx(I+AT)\xi(t-T)+BTe(t-T)
\end{align*}
which is simple but can result in an unstable compensator as $s=\lambda \rightarrow z=1+\lambda T$.\\
%
\ptitle{Numerical Emulation: Backward Difference}

\noindent\begin{align*}
    \xi(t)-\xi(t-T) & \approx T\cdot\dot{\xi}(T)                   \\
                    & \text{yields}                                \\
    \xi(t)          & \approx(I-AT)^{-1}\xi(t-T)+(I-AT)^{-1}BTe(t)
\end{align*}
which avoids instability as $s=\lambda \rightarrow z=(1-\lambda T)^{-1}$. This transforms poles into a disk smaller than the unit disk. DT poles cannot get negative real part.\\
\ptitle{Numerical Emulation: Trapezoidal Rule}

\noindent\begin{align*}
    \xi(t)-\xi(t-T) & \approx \frac T2\left(\dot{\xi}(t)+\dot{\xi}(t-T)\right) \\
                    & \text{yields}                                            \\
    \xi(t)          & \approx(I-AT/2)^{-1}(I+AT/2)\xi(t-T)                     \\
                    & +(I-AT/2)^{-1}BT\frac{e(t)+e(t-T)}2
\end{align*}
which maps the left half plane to the unit disk.