\section{Stability and Performance Robustness}
One main achievement of computing the $\mathcal{H}_\infty$ norm is to access stability and performance robustness and achieve robust control.
\subsection{The Small Gain Theorem (SGT)}
The (unstructured) SGT attempts to access interconnection stability. In contrast to the structured SGT, the SGT does not make any assumptions on the structure of the uncertainty matrix.

\subsubsection{Sufficient Condition}

The interconnection of two \textbf{stable} (MIMO) systems, with transfer functions $\mathbf{G_1}(s)$ and $\mathbf{G_2}(s)$, is stable if
\begin{equation*}
    \left\|\mathbf{G_1}\right\|_{\mathcal{H}_\infty}\cdot\left\|\mathbf{G_2}\right\|_{\mathcal{H}_\infty}<1
\end{equation*}
\begin{center}
    \includegraphics[width = 0.5\linewidth]{SGT_interconn.png}
\end{center}

\ptitle{Remarks}
\begin{itemize}
    \item There could be some systems that violate this condition but still lead to a stable feedback interconnection.
    \item \textbf{Any} norm fulfilling the submultiplicative property can be used for the SGT!
    \item The small-gain theorem also applies to non-linear, time-varying systems.
    \item To have sufficiency, $(\mathbf{I}-\mathbf{G_1G_2})$ must not have zeros in the RHP.
\end{itemize}

\subsubsection{Necessary and Sufficient Condition}
Given a feedback interconnection of $\mathbf{G}$ and an unknown system $\Delta$ where
\begin{itemize}
    \item $\Delta$ is stable
    \item and $\left\|\Delta\right\|_{\mathcal{H}_\infty}<1$
\end{itemize}
the FB interconnection of $\mathbf{G}$ and $\Delta$ is \textbf{guaranteed} to be \textbf{stable iff}

\begin{equation*}
    {\left\|\mathbf{G}\right\|}_{\mathcal{H}_{\infty}}<1
\end{equation*}
\begin{itemize}
    \item Otherwise, one can find a $\Delta$ so that $(\mathbf{I}-\mathbf{G}\Delta)$ singular.
    \item Note that one only makes the conservative assumptions: $\Delta$ stable and $\left\|\Delta\right\|_{\mathcal{H}_\infty}<1$
          \begin{itemize}
            %   \item Under these assumptions one has an ``iff'' condition.
              \item In fact, one can sometimes make more assumptions (see SSV).
          \end{itemize}
\end{itemize}

\subsection{Modelling Uncertainty}
\begin{center}
    \includegraphics[width = 0.5\linewidth]{uncertainty.png}
\end{center}
To take uncertainty into account, one first defines an \textbf{uncertainty model}, consisting of
\begin{enumerate}
    \item A nominal model $\mathbf{P}$
    \item A set of models that is guaranteed to contain the system uncertainty, and is easier to handle
\end{enumerate}
and then designs a control system that meets the stability and performance specifications for all possible models in the \textit{uncertainty model}.

\newpar{}
\ptitle{Assumptions on Uncertainty}

In the following uncertainty models one assumes that $\Delta$
\begin{itemize}
    \item is minimum-phase
    \item does not cancel unstable poles of the nominal system
    \item $|\Delta(j\omega)|<1,\quad\forall\omega$
\end{itemize}
and that there is a high-pass transfer function $W$ called \textbf{frequency weight} so that the total uncertainty is given by
\begin{equation*}
    \mathbf{W}(s)\Delta(s)
\end{equation*}

\subsubsection{Common Uncertainty Models}
\ptitle{Additive Uncertainty}
\begin{equation*}
    \tilde{\mathbf{P}}(s)=\mathbf{P}(s)+\mathbf{W}(s)\Delta(s)
\end{equation*}
\begin{center}
    \includegraphics[width = 0.75\linewidth]{additive_uncertainty.png}
\end{center}

\ptitle{Multiplicative Uncertainty}
\begin{equation*}
    \tilde{\mathbf{P}}(s)=\mathbf{P}(s)(1+\mathbf{W}(s)\Delta(s))
\end{equation*}
\begin{center}
    \includegraphics[width = 0.75\linewidth]{multiplicative_uncertainty.png}
\end{center}
\ptitle{Feedback Uncertainty}
\begin{equation*}
    \tilde{\mathbf{P}}(s)={(\mathbf{I}-\mathbf{P}(s)\mathbf{W}(s)\Delta(s))}^{-1}\mathbf{P}(s)
\end{equation*}
\begin{center}
    \includegraphics[width = 0.75\linewidth]{fb_uncertainty.png}
\end{center}

\subsection{Robust Stability}
\begin{itemize}
    \item Equivalent to SISO bode plot obstacles (now obstacles are given by singular values and in time domain)    % TODO: Why in time domain?
    \item Now we impose constraints on the $\mathcal{H}_{\infty}$ norm
    \item \textbf{Caution}: depending on the convention the block diagrams vary and some matrices in a specific $\mathbf{G}$ will be permuted to the forms shown below.
\end{itemize}

\ptitle{General Procedure}

\begin{center}
    \includegraphics[width = 0.35\linewidth]{general_robust_fb.png}
\end{center}
Given an uncertainty and a controller designed for the nominal plant $\mathbf{P}$ one computes the transfer function without the system $\boldsymbol{\Delta}$ and then applies the SGT. The transfer function $\mathbf{G}$ to be found is from the output of $\boldsymbol{\Delta}$ to it's input i.e.\ find:
\begin{equation*}
    \mathbf{G}_{BA}=\mathbf{BA}^{-1}
\end{equation*}

\ptitle{Additive Uncertainty}
\begin{center}
    \includegraphics[width = 0.75\linewidth]{robust_additive.png}
\end{center}
The CL system is robustly stable iff
\begin{equation*}
    \|\mathbf{G}(s)\|_{\mathcal{H}_\infty}=\|-{(\mathbf{I}+\mathbf{C}\mathbf{P})}^{-1}\mathbf{C}\mathbf{W}\|_{\mathcal{H}_\infty}<1
\end{equation*}
Remember: this is a necessary and sufficient condition.

\ptitle{Multiplicative Uncertainty}
\begin{center}
    \includegraphics[width = 0.75\linewidth]{robust_multiplicative.png}
\end{center}
The CL system is robustly stable iff
\begin{equation*}
    \|\mathbf{G}(s)\|_{\mathcal{H}_\infty}=\|-{(\mathbf{I}+\mathbf{CP})}^{-1}\mathbf{CPW}\|_{\mathcal{H}_\infty}<1
\end{equation*}

\ptitle{Feedback Uncertainty}
\begin{center}
    \includegraphics[width = 0.75\linewidth]{robust_feedback.png}
\end{center}
The CL system is robustly stable iff
\begin{equation*}
    \|\mathbf{G}(s)\|_{\mathcal{H}_\infty}=\|\mathbf{W}{(\mathbf{I}+\mathbf{PC})}^{-1}\mathbf{P}\|_{\mathcal{H}_\infty}<1
\end{equation*}

\subsection{Robust Performance}
The “robust performance” problem can be formulated in a way that matches the “robust stability” problem.
\begin{itemize}
    \item Isolate the two signals between which amplification is unwanted.
    \item Apply the SGT to the rest of the system as done for robust control.
\end{itemize}

\ptitle{Disturbance at Output}

\begin{center}
    \includegraphics[width = 0.75\linewidth]{robust_performance_fb.png}
\end{center}
\begin{itemize}
    \item We don't want the energy (induced 2-norm) of disturbance $\xi$ to be amplified at the output.
    \item Therefore we want to limit the gain from $\xi$ to $y$
          \begin{itemize}
              \item Instead of unwanted amplification in the FB path we now look at $\xi \rightarrow y$.
          \end{itemize}
\end{itemize}

Applying the SGT (assuming $\boldsymbol{\Delta}=\mathbf{G}_{y\xi}$) yields
\begin{equation*}
    \|\mathbf{G}_{y\xi}\|_{\mathcal{H}_\infty}=\|{(\mathbf{I}+\mathbf{PC})}^{-1}\mathbf{W}\|_{\mathcal{H}_\infty}<1
\end{equation*}

\subsection{Robust Disturbance Rejection}
\begin{center}
    \includegraphics[width = \linewidth]{combined_robustness.png}
\end{center}
We want to consider robust stability and robust performance \textbf{simultaneously}. To do so we need
\begin{itemize}
    \item An \textbf{uncertainty block}
    \item A \textbf{transfer function matrix} describing all uncertainty input-output relations
\end{itemize}
We model an uncertainty block by
\begin{equation*}\label{diag_unc_block}
    \begin{bmatrix}
        w_1 \\
        w_2
    \end{bmatrix}
    =\underbrace{\begin{bmatrix}
            \Delta_1 & 0        \\
            0        & \Delta_2
        \end{bmatrix}}_{\boldsymbol{\Delta}}
    \begin{bmatrix}z_1 \\
        z_2
    \end{bmatrix}\qquad
    \begin{matrix}
        \mathrm{stability} \\
        \mathrm{robustness}
    \end{matrix}
\end{equation*}
And the transfer function matrix for the SGT by
\begin{align*}
    \mathbf{M} & =
    \begin{bmatrix}
        G_{{z_{1}w_{1}}} & G_{{z_{1}w_{2}}} \\
        G_{{z_{2}w_{1}}} & G_{{z_{2}w_{2}}}
    \end{bmatrix}                                                                                                                                                                                                                  \\
               & =\begin{bmatrix}
                      -\mathbf{W}_1\mathbf{P}_0\mathbf{K}{(\mathbf{I}+\mathbf{P}_0\mathbf{K})}^{-1} & -\mathbf{W}_1 \mathbf{P}_0\mathbf{K}{(\mathbf{I}+\mathbf{P}_0\mathbf{K})}^{-1} \\
                      \mathbf{W}_2{(\mathbf{I}+\mathbf{P}_0\mathbf{K})}^{-1}                        & \mathbf{W}_2{(\mathbf{I}+\mathbf{P}_0\mathbf{K})}^{-1}
                  \end{bmatrix}
\end{align*}
\subsubsection{Robustness Assessment}
Robustness can now be accessed by applying the SGT to $\mathbf{M}$
\begin{equation*}
    \left\|\mathbf{M}\right\|_{\mathcal{H}_\infty}<1
\end{equation*}
to achieve robust \textbf{disturbance rejection}.

\newpar{}
\ptitle{Remarks}
\begin{itemize}
    \item Note that we ignored the diagonal structure of $\boldsymbol{\Delta}$ which makes our assumptions too conservative (see SSV for less conservatism).
    \item Also note that $\mathbf{M}$ has rank 1. In this case one can calculate the SSV exactly (see below).
\end{itemize}

\subsection{Structured Singular Value (SSV)}
The condition from the \textit{unstructured SGT} is a conservative assumption as the SGT could be applied to an arbitrary $\boldsymbol{\Delta}$.
For example, as the $\boldsymbol{\Delta}$ from~\ref{diag_unc_block} has block-diagonal structure, less conservative robustness conditions could be applied.

\newpar{}
\ptitle{Definition of SSV}

The SSV is defined with respect to a \textbf{class of perturbations} $\mathbb{D}$ as the inverse value of the smallest $\sigma_{\max}$ making $\mathbf{M}$ singular:
\begin{equation*}
    \mu(\mathbf{M}):=\frac1{\inf\{\sigma_{\max}(\boldsymbol{\Delta}):\det(1-M\boldsymbol{\Delta})=0\}},\quad\boldsymbol{\Delta}\in\mathbb{D}
\end{equation*}
\begin{itemize}
    \item Convention: If $\det(I-M\boldsymbol{\Delta})\neq0,\forall\boldsymbol{\Delta}\in\mathbb{D}\text{, then }\mu(M)=0$.
    \item If $\boldsymbol{\Delta}$ is diagonal and $\mathbf{M}$ has rank one the SSV can be computed analytically.
\end{itemize}

\ptitle{Properties of the SSV}
\begin{itemize}
    \item $\mu(\mathbf{M})\geq0$
    \item If $\mathbb{D}$ is arbitrary: $\mu(\mathbf{M})=\|\mathbf{M}\|_{\mathcal{H}_\infty}$ (unstructured case)
    \item If $\mathbb{D}=\{\lambda I:\lambda\in\mathbb{C}\}$: $\mu(\mathbf{M})=\rho(\mathbf{M})$ (spectral radius = largest eigenvalue, easy to compute)
    \item If $\mathbb{D}$ is the set of diagonal (complex) matrices then
          \begin{equation*}
              \mu(\mathbf{M})=\mu(\mathbf{D}^{-1}\mathbf{MD})
          \end{equation*}
          for any invertible diagonal (complex) scaling matrix $\mathbf{D}$ and
          \begin{equation*}
              \rho(\mathbf{M})\leq\mu(\mathbf{M})\leq\inf_{\mathbf{D}}\sigma_{\max}(\mathbf{D}^{-1}\mathbf{MD})\leq\sigma_{\max}(\mathbf{M})
          \end{equation*}
\end{itemize}

\subsubsection{Accessing Stability Using SSV}
\ptitle{Stability Condition}

The $\mathbf{M}-\boldsymbol{\Delta}$ FB system is stable for all $\boldsymbol{\Delta}\in \mathbb{D}$, $\left\|\boldsymbol{\Delta}\right\|_{\mathcal{H}_\infty}<1$ iff
\begin{equation*}
    \mu(\mathbf{M}(j\omega))\leq1,\quad\forall\omega\in\mathbb{R}
\end{equation*}
which is less conservative than the SGT.

\paragraph{Rank-One Problems}
In general, there is no closed-form method for computing the SSV (must resort to, e.g., estimating upper bounds (e.g., by scaling matrices)).\\
However, when
\begin{itemize}
    \item $\boldsymbol{\Delta}$ is a diagonal (complex) matrix,
    \item and $\mathbf{M}$ is a rank 1 matrix, i.e.,
          \begin{equation*}
              \mathbf{M}=\mathbf{ab^H}
          \end{equation*}
          for some complex vectors $\mathbf{a},\mathbf{b}\in \mathbb{C}^n$
\end{itemize}
then one can compute the structured singular value exactly.\\
One has
\begin{equation*}
    \det(\mathbf{I}-\mathbf{M}\boldsymbol{\Delta})=1-\sum_{i}\mathbf{a}_{i}\mathbf{b}_{i}^{H}\delta_{i}
\end{equation*}
and since $\sigma_{\max}(\boldsymbol{\Delta})=\max_i|\delta_i|$ one has
\begin{equation*}
    \mu(\mathbf{M})=\sum_i|\mathbf{a}_i \mathbf{b}_i^H|
\end{equation*}

\newpar{}
\ptitle{Remarks}
\begin{itemize}
    \item Given the mentioned conditions one can express $\mathbf{M}$ as the product of two vectors.
    \item One can then easily calculate the SSV from these vectors (instead of from $\mathbf{M}$).
\end{itemize}

\begin{examplesection}[Example: Scalar Combined Robustness]
    The aforementioned conditions are fulfilled as $\boldsymbol{\Delta}$ diagonal and $\mathbf{M}$ has rank 1. We have
    \begin{equation*}
        \mathbf{M}=\mathbf{ab^H}=\begin{bmatrix}
            -\frac{\mathbf{W}_1\mathbf{P}_0\mathbf{K}}{1+\mathbf{P}_0\mathbf{K}} \\
            \frac{\mathbf{W}_2}{1+\mathbf{P}_0\mathbf{K}}
        \end{bmatrix}
        \begin{bmatrix}
            1 & 1
        \end{bmatrix}
    \end{equation*}
    which yields
    \begin{align*}
        \mu(\mathbf{M}(j\omega)) & =\left|\frac{\mathbf{W}_1\mathbf{P}_0\mathbf{K}}{1+\mathbf{P}_0\mathbf{K}}(j\omega)\right|+\left|\frac{\mathbf{W}_2}{1+\mathbf{P}_0\mathbf{K}}(j\omega)\right|\leq1 \\
                                 & =|\mathbf{W}_1 \mathbf{L}(j\omega)|+|\mathbf{W}_2(j\omega)|\leq|1+\mathbf{L}(j\omega)|\quad\forall\omega\in\mathrm{R}
    \end{align*}
    where typically
    \begin{itemize}
        \item $\mathbf{L}=\mathbf{P}_0 \mathbf{K}$
        \item $\mathbf{W}_1$ is a high-pass
        \item $\mathbf{W}_2$ is a low-pass
        \item $|\mathbf{L}|$ is a low-pass (low-pass assumption for physical systems)
    \end{itemize}
    Therefore, one has for low frequencies
    \begin{equation*}
        |\mathbf{W}_1(j\omega)|+\frac{|\mathbf{W}_2(j\omega)|}{|\mathbf{L}(j\omega)|}\leq1
    \end{equation*}
    \begin{equation*}
        |\mathbf{L}(j\omega)|\geq\frac{|\mathbf{W}_2(j\omega)|}{1-|\mathbf{W}_1(j\omega)|}\approx|\mathbf{W}_2(j\omega)|
    \end{equation*}
    and for high frequencies
    \begin{equation*}
        |\mathbf{W}_1(j\omega)||\mathbf{L}(j\omega)|+|\mathbf{W}_2(j\omega)|\leq1
    \end{equation*}
    \begin{equation*}
        |\mathbf{L}(j\omega)|\leq\frac{1-|\mathbf{W}_2(j\omega)|}{|\mathbf{W}_1(j\omega)|}\approx\frac1{|\mathbf{W}_1(j\omega)|}
    \end{equation*}
    which corresponds to the CS1 Bode obstacles.\\
    One can visualize the condition by an advanced Nyquist criterion.
    \begin{center}
        \includegraphics[width = 0.7\linewidth]{chad_nyquist.png}
    \end{center}
\end{examplesection}


