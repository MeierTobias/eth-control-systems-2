\subsection{Stability and Performance Robustness}
One main achievement of computing the $\mathcal{H}_\infty$ norm is to access stability and performance robustness and achieve robust control.
\subsubsection{The Small Gain Theorem (SGT)}
The (unstructured) SGT attempts to access interconnection stability. In contrast to the structured SGT, the SGT does not make any assumptions on the structure of the uncertainty matrix.

\paragraph{Sufficient Condition}

The interconnection of two stable (MIMO) systems, with transfer functions $\mathbf{G_1}(s)$ and $\mathbf{G_2}(s)$, is stable if
\begin{equation*}
    \left\|\mathbf{G_1}\right\|_{\mathcal{H}_\infty}\cdot\left\|\mathbf{G_2}\right\|_{\mathcal{H}_\infty}<1
\end{equation*}

\ptitle{Remarks}
\begin{itemize}
    \item There could be some systems that violate this condition but still lead to a stable feedback interconnection.
    \item The small-gain theorem also applies to non-linear, time-varying systems.
    \item To have sufficiency, $(\mathbf{I}-\mathbf{G_1G_2})$ must not have zeros in the RHP.
\end{itemize}

\paragraph{Necessary and Sufficient Condition}
Given a feedback interconnection of $\mathbf{G}$ and an unknown system $\Delta$ where
\begin{itemize}
    \item $\Delta$ is stable
    \item and $\left\|\Delta\right\|_{\mathcal{H}_\infty}<1$
\end{itemize}
the FB interconnection of $\mathbf{G}$ and $\Delta$ is \textbf{stable iff}
\begin{equation*}
    \left\|\mathbf{G}\right\|_{\mathcal{H}_\infty}<1
\end{equation*}
Otherways, one can find a $\Delta$ so that $(\mathbf{I}-\mathbf{G\Delta})$ singular.\\

\subsubsection{Modelling Uncertainty}
\begin{center}
    \includegraphics[width = 0.5\linewidth]{uncertainty.png}
\end{center}
To take uncertainty into account, one first defines an \textbf{uncertainty model}, consisting of
\begin{enumerate}
    \item A nominal model $P$
    \item A set of models that is guaranteed to contain the system uncertainty, and is easier to handle
\end{enumerate}
and then designs a control system that meets the stability and performance specifications not only for $P$, but also for all other possible models in the uncertainty model.

\ptitle{Assumptions on Uncertainty}

In the following uncertainty models one assumes that $\Delta$
\begin{itemize}
    \item is minimum-phase
    \item does not cancle unstable poles of the nominal system
    \item $|\Delta(j\omega)|<1,\quad\forall\omega$
\end{itemize}
and that there is a high-pass transfer function $W$ called \textbf{frequency weight} so that the total uncertainty is given by
\begin{equation*}
    W(s)\Delta(s)
\end{equation*}

\paragraph{Common Uncertainty Models}
\ptitle{Additive Uncertainty}
\begin{equation*}
    \tilde{P}(s)=P(s)+W(s)\Delta(s)
\end{equation*}
\begin{center}
    \includegraphics[width = 0.75\linewidth]{additive_uncertainty.png}
\end{center}

\ptitle{Multiplicative Uncertainty}
\begin{equation*}
    \tilde{P}(s)=P(s)(1+W(s)\Delta(s))
\end{equation*}
\begin{center}
    \includegraphics[width = 0.75\linewidth]{multiplicative_uncertainty.png}
\end{center}
\ptitle{Feedback Uncertainty}
\begin{equation*}
    \tilde{P}(s)=(I-P(s)W(s)\Delta(s))^{-1}P(s)
\end{equation*}
\begin{center}
    \includegraphics[width = 0.75\linewidth]{fb_uncertainty.png}
\end{center}

\subsubsection{Robust Stability}
\begin{itemize}
    \item Equivalent to SISO bode plot obstacles (now obstacles are given by singular values)
    \item Now we impose constraints on the $\mathcal{H}_{\infty}$ norm
\end{itemize}

\ptitle{Additive Uncertainty}
\begin{center}
    \includegraphics[width = 0.75\linewidth]{robust_additive.png}
\end{center}
Given an additive uncertainty and a controller designed for the nominal plant $\mathbf{P}$ one computes the transfer function without $\Delta$ and applies the SGT. The CL system is then robustly stable iff
\begin{equation*}
    \|\mathbf{G}(s)\|_{\mathcal{H}_\infty}=\|-{(\mathbf{I}+\mathbf{C}(s)\mathbf{P}(s))}^{-1}\mathbf{C}(s)\mathbf{W}(s)\|_{\mathcal{H}_\infty}<1
\end{equation*}
Remember: this is a necessary and sufficient condition.

\ptitle{Multiplicative Uncertainty}
%%% TBD %%%
% @Tobi: hesch du chöne mitschribe vo de Wandtafle?
%%% TBD %%%

\subsubsection{Robust Performance}
The “robust performance” problem can be formulated in a way that matches the “robust stability” problem.
\begin{itemize}
    \item Isolate the two signals between which amplification is unwanted.
    \item Apply the SGT to the rest of the system as done for robust control.
\end{itemize}

\ptitle{Disturbance at Output}

\begin{center}
    \includegraphics[width = 0.75\linewidth]{robust_performance_fb.png}
\end{center}
\begin{itemize}
    \item We don't want the energy of disturbance $\xi$ to be amplified at the output
    \item Therefore we want to limit the gain from $\xi$ to $y$
\end{itemize}

Applying the SGT (assuming $\Delta=G_{y\xi}$) yields
\begin{equation*}
    \|G_{y\xi}\|_{\mathcal{H}_\infty}=\|(I+PC)^{-1}W\|_{\mathcal{H}_\infty}<1
\end{equation*}

\subsubsection{Combined Robustness}
\begin{center}
    \includegraphics[width = 0.75\linewidth]{combined_robustness.png}
\end{center}
We want to consider robust stability and robust performance simultaneously. Hence, we model an \textbf{uncertainty block}
\begin{equation*}
    \begin{bmatrix}
        w_1 \\
        w_2
    \end{bmatrix}
    =\underbrace{\begin{bmatrix}
            \Delta_1 & 0        \\
            0        & \Delta_2
        \end{bmatrix}}_{\boldsymbol{\Delta}}
    \begin{bmatrix}z_1 \\
        z_2
    \end{bmatrix}
\end{equation*}

\paragraph{Applying the SGT}

The transfer function matrix for the SGT is given by
\begin{align*}
    \mathbf{M} & =
    \begin{bmatrix}
        G_{{z_{1}w_{1}}} & G_{{z_{1}w_{2}}} \\
        G_{{z_{2}w_{1}}} & G_{{z_{2}w_{2}}}
    \end{bmatrix}                     \\
               & =\begin{bmatrix}
                      -W_1PK(I+P_0K)^{-1} & -W_1PK(I+P_0K)^{-1} \\
                      W_2(I+P_0K)^{-1}    & W_2(I+P_0K)^{-1}
                  \end{bmatrix}
\end{align*}
%%% TBD %%%
% Why are the delta blocks of the "other" BSB parts not in the TF matrix?
%%% TBD %%%
Therefore, we impose
\begin{equation*}
    \left\|\mathbf{M}\right\|_{\mathcal{H}_\infty}<1
\end{equation*}
However, this is a conservative assumption as the SGT could be applied to an arbitrary $\boldsymbol{\Delta}$. Therefore, we adjust the condition to more specific uncertainty matrices.

\paragraph{Applying the Structured Singular Value (SSV)}

\ptitle{Definition of SSV}

The SSV is defined with respect to a \textbf{class of perturbations} $\mathrm{D}$ as
\begin{equation*}
    \mu(\mathbf{M}):=\frac1{\inf\{\sigma_{\max}(\boldsymbol{\Delta}):\det(1-M\boldsymbol{\Delta})=0\}},\quad\boldsymbol{\Delta}\in\mathbb{D}
\end{equation*}
\begin{itemize}
    \item By convention, if $\det(I-M\boldsymbol{\Delta})\neq0,\forall\boldsymbol{\Delta}\in\mathbb{D}\text{, then }\mu(M)=0$.
    \item If $\boldsymbol{\Delta}$ is diagonal and $M$ has rank one the SSV can be computed analytically.
\end{itemize}

\ptitle{Properties of the SSV}
\begin{itemize}
    \item $\mu(\mathbf{M})\geq0$
    \item If $\mathbb{D}$ is arbitrary: $\mu(\mathbf{M})=\|\mathbf{M}\|_{\mathcal{H}_\infty}$ (unstructured case)
    \item If $\mathbb{D}=\{\lambda I:\lambda\in\mathbb{C}\}$: $\mu(\mathbf{M})=\rho(\mathbf{M})$ (spectral radius, largest eigenvalue)
    \item If $\mathbb{D}$ diagonal (complex): $\mu(\mathbf{M})=\mu(\mathbf{D}^{-1}\mathbf{MD})$ for any invertible $\mathbf{D}$ and
    \item \begin{equation*}
              \rho(\mathbf{M})\leq\mu(\mathbf{M})\leq\inf_{\mathbf{D}}\sigma_{\max}(\mathbf{D}^{-1}\mathbf{MD})\leq\sigma_{\max}(\mathbf{M})
          \end{equation*}
\end{itemize}


\ptitle{Stability Condition}

The $\mathbf{M}-\boldsymbol{\Delta}$ FB system is stable for all $\boldsymbol{\Delta}\in D$, $\left\|\boldsymbol{\Delta}\right\|_{\mathcal{H}_\infty}<1$ iff
\begin{equation*}
    \mu(\mathbf{M}(j\omega))\leq1,\quad\forall\omega\in\mathbb{R}
\end{equation*}

\ptitle{Scalar Combined Robustness}

\begin{equation*}
    \mathbf{M}=\begin{bmatrix}
        -\frac{W_1P_0K}{1+P_0K} \\
        \frac{W_2}{1+P_0K}
    \end{bmatrix}
    \begin{bmatrix}
        1 & 1
    \end{bmatrix}
\end{equation*}
yields

\begin{align*}
    \mu(\mathbf{M}(j\omega)) & =\left|\frac{W_1PK}{1+P_0K}(j\omega)\right|+\left|\frac{W_2}{1+P_0K}(j\omega)\right|\leq1 \\
                             & =|W_1L(j\omega)|+|W_2(j\omega)|\leq|1+L(j\omega)|\quad\forall\omega\in\mathrm{R}
\end{align*}
where
\begin{itemize}
    \item $L=P_0K$
    \item $W_1$ is a high-pass
    \item $W_2$ is a low-pass
    \item $|L|$ is a low-pass (low-pass assumption for physical systems)
\end{itemize}
Therefore, one has for low frequencies
\begin{equation*}
    |W_1(j\omega)|+\frac{|W_2(j\omega)|}{|L(j\omega)|}\leq1
\end{equation*}
\begin{equation*}
    |L(j\omega)|\geq\frac{|W_2(j\omega)|}{1-|W_1(j\omega)|}\approx|W_2(j\omega)|
\end{equation*}
and for high frequencies
\begin{equation*}
    |W_1(j\omega)||L(j\omega)|+|W_2(j\omega)|\leq1
\end{equation*}
\begin{equation*}
    |L(j\omega)|\leq\frac{1-|W_2(j\omega)|}{|W_1(j\omega)|}\approx\frac1{|W_1(j\omega)|}
\end{equation*}
One can look at the condition with an advanced Nyquist condition.
\begin{center}
    \includegraphics[width = 0.5\linewidth]{chad_nyquist.png}
\end{center}

