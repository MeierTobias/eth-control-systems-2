\section{Appendix}
\subsection{Matrices}
\subsubsection{Matrix Properties}\label{app:mat_prop}

\ptitle{Orthogonal Matrix}
\noindent\begin{equation*}
    \mathbf{U}^{\mathsf{T}}\mathbf{U}=\mathbf{UU}^{\mathsf{T}}=\mathbf{I}\quad \mathbf{U}\in \mathbb{R}^{n\times n}
\end{equation*}

\ptitle{Unitary Matrix}
\noindent\begin{equation*}
    \mathbf{U}^{\mathsf{H}}\mathbf{U}=\mathbf{UU}^{\mathsf{H}}=\mathbf{I}\quad \mathbf{U}\in \mathbb{R}^{n\times n}
\end{equation*}
\begin{itemize}
    \item preserves the euclidean norm:
          \noindent\begin{equation*}
              \|\mathbf{Ux}\| _2=\|\mathbf{x}\|_2 \quad \forall \mathbf{x} \in \mathbb{C}^n
          \end{equation*}
\end{itemize}

\ptitle{Hermitian Matrix}
\noindent\begin{equation*}
    \mathbf{S}=\mathbf{S}^{\mathsf{H}}
\end{equation*}
For any
\noindent\begin{equation*}
    \mathbf{A} \in \mathbb{R}^{m\times n}
\end{equation*}
both
\noindent\begin{align*}
    \mathbf{A^H A} & \in \mathbb{R}^{n\times n} \\
    \mathbf{A A^H} & \in \mathbb{R}^{m\times m}
\end{align*}
are hermitian, positive semi-definite and their eigenvalues are real and non-negative.
\begin{itemize}
    \item For every hermitian matrix $\mathbf{S}$ exists a unitary matrix $\mathbf{U}$ s.t. $\mathbf{U}^{\mathsf{H}}\mathbf{SU}$ is a diagonal matrix.
          \begin{itemize}
              \item In other words, unitary matrices can diagonalize hermitian matrices which is what one uses for SVD.
          \end{itemize}
\end{itemize}

\newpar{}
\ptitle{Matrix Exponential Function}
\begin{equation*}
    e^{\mathbf{X}} = \sum_{k=0}^{\infty}\frac{1}{k!}\mathbf{X^k}
\end{equation*}

\subsubsection{Transpose, Adjoint and Inverse}
\noindent\begin{align*}
    \mathbf{A}^\mathsf{H}                  & = {(\mathbf{A}^\mathsf{T})}^*                   \\[.75em]
    {({\mathbf{A}}^\mathsf{T})}^\mathsf{T} & = \mathbf{A}                                    \\
    {({\mathbf{A}}^\mathsf{H})}^\mathsf{H} & = \mathbf{A}                                    \\
    {({\mathbf{A}}^{-1})}^{-1}             & = \mathbf{A}                                    \\[.75em]
    {(\lambda\mathbf{A})}^\mathsf{T}       & = \lambda\mathbf{A}^\mathsf{T}                  \\
    {(\lambda\mathbf{A})}^\mathsf{H}       & = \lambda\mathbf{A}^\mathsf{H}                  \\
    {(\lambda\mathbf{A})}^{-1}             & = \lambda^{-1}\mathbf{A}^{-1}                   \\[.75em]
    {(\mathbf{A}+\mathbf{B})}^\mathsf{T}   & = \mathbf{A}^\mathsf{T} + \mathbf{B}^\mathsf{T} \\
    {(\mathbf{A}+\mathbf{B})}^\mathsf{H}   & = \mathbf{A}^\mathsf{H} + \mathbf{B}^\mathsf{H} \\[.75em]
    {(\mathbf{A}\mathbf{B})}^\mathsf{T}    & = \mathbf{B}^\mathsf{T}\mathbf{A}^\mathsf{T}    \\
    {(\mathbf{A}\mathbf{B})}^\mathsf{H}    & = \mathbf{B}^\mathsf{H}\mathbf{A}^\mathsf{H}    \\
    {(\mathbf{A}\mathbf{B})}^{-1}          & = \mathbf{B}^{-1} \mathbf{A}^{-1}               \\[.75em]
    \det(\mathbf{A}^\mathsf{T})            & = \det(\mathbf{A})                              \\
    \det(\mathbf{A}^\mathsf{H})            & = {\det(\mathbf{A})}^*                          \\
    \det(\mathbf{A}^{-1})                  & = {\det(\mathbf{A})}^{-1}                       \\[.75em]
    \mathrm{rank}{(\mathbf{A}^\mathsf{H})} & = \mathrm{rank}(\mathbf{A})                     \\
    \mathrm{rank}{(\mathbf{A}^{-1} )}      & = \mathrm{rank}(\mathbf{A})
\end{align*}

\ptitle{\textbf{A} is symmetric}
\noindent\begin{align*}
    {\mathbf{A}}^\mathsf{T} & = \mathbf{A}   \\
    {\mathbf{A}}^\mathsf{H} & = \mathbf{A}^*
\end{align*}

\ptitle{\textbf{A} is invertible}
\noindent\begin{align*}
    {(\mathbf{A}^\mathsf{T})}^{-1} & = {(\mathbf{A}^{-1})}^\mathsf{T} \\
    {(\mathbf{A}^\mathsf{H})}^{-1} & = {(\mathbf{A}^{-1})}^\mathsf{H}
\end{align*}

\ptitle{\textbf{A} is square}
\noindent\begin{align*}
    \mathrm{eigvals}(\mathbf{A}^\mathsf{T}) & = \mathrm{eigvals}(\mathbf{A})        \\
    \mathrm{eigvals}(\mathbf{A}^\mathsf{H}) & = \mathrm{eigvals}{(\mathbf{A})}^*    \\
    \mathrm{eigvals}(\mathbf{A}^{-1})       & = \mathrm{eigvals}{(\mathbf{A})}^{-1}
\end{align*}

\ptitle{Blocks}
\noindent\begin{equation*}
    \begin{pmatrix}
        A_{11} & \cdots & A_{1n} \\
        \vdots &        & \vdots \\
        A_{m1} & \cdots & A_{mn}
    \end{pmatrix}^\mathsf{H}=
    \begin{pmatrix}
        A_{11}^\mathsf{H} & \cdots & A_{m1}^\mathsf{H} \\
        \vdots            &        & \vdots            \\
        A_{1n}^\mathsf{H} & \cdots & A_{nm}^\mathsf{H}
    \end{pmatrix}
\end{equation*}

\subsubsection{Specific Solutions}
\ptitle{Adjoint of a $3\times3$ matrix}
\begin{small}
    \noindent\begin{align*}
        \mathrm{adj}(\mathbf{A}) & = \begin{bmatrix}
                                         +\begin{vmatrix}a_{22}&a_{23}\\a_{32}&a_{33}\end{vmatrix} & -\begin{vmatrix}a_{12}&a_{13}\\a_{32}&a_{33}\end{vmatrix} & +\begin{vmatrix}a_{12}&a_{13}\\a_{22}&a_{23}\end{vmatrix} \\\\
                                         -\begin{vmatrix}a_{21}&a_{23}\\a_{31}&a_{33}\end{vmatrix} & +\begin{vmatrix}a_{11}&a_{13}\\a_{31}&a_{33}\end{vmatrix} & -\begin{vmatrix}a_{11}&a_{13}\\a_{21}&a_{23}\end{vmatrix} \\\\
                                         +\begin{vmatrix}a_{21}&a_{22}\\a_{31}&a_{32}\end{vmatrix} & -\begin{vmatrix}a_{11}&a_{12}\\a_{31}&a_{32}\end{vmatrix} & +\begin{vmatrix}a_{11}&a_{12}\\a_{21}&a_{22}\end{vmatrix}
                                     \end{bmatrix} \\
                                 & = \begin{bmatrix}
                                         a_{22}a_{33}-a_{23}a_{32} & a_{13}a_{32}-a_{12}a_{33} & a_{12}a_{23}-a_{13}a_{22} \\
                                         a_{23}a_{31}-a_{21}a_{33} & a_{11}a_{33}-a_{13}a_{31} & a_{13}a_{21}-a_{11}a_{23} \\
                                         a_{21}a_{32}-a_{22}a_{31} & a_{12}a_{31}-a_{11}a_{32} & a_{11}a_{22}-a_{12}a_{21} \\
                                     \end{bmatrix}
    \end{align*}
\end{small}

\ptitle{Determinant}
\noindent\begin{equation*}
    \det(A)=\sum_{j=1}^n \underbrace{{(-1)}^{i+j}}_{\textsf{checkboard pattern}}\cdot a_{i,j}\cdot\underbrace{m_{i,j}}_{\textsf{minor: |submatrix of }\mathbf{A}|}
\end{equation*}
\textbf{Remark} Any row or column can be chosen (choose wisely: 0)

\newpar{}
\ptitle{Inversion}

\begin{align*}
    \mathbf{A}^{-1} & = \frac{\text{adj}(\mathbf{A})}{\text{det}(\mathbf{A})} \\
    \mathbf{A}^{-1} & = \frac{1}{a_{11}a_{22} - a_{12}a_{21}}
    \begin{bmatrix}
        a_{22}  & - a_{12} \\
        -a_{21} & a_{11}
    \end{bmatrix}
\end{align*}