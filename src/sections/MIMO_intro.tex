\section{Introduction to MIMO Systems}
\subsection{Fundamentals of MIMO Systems}

\ptitle{Comparison to SISO}
MIMO Systems:
\begin{itemize}
    \item have TF matrices
    \item impose the concept of directionality
    \item can adapt many SISO ideas but not all (poles, zeros, system response phase measurement)
\end{itemize}

\subsubsection{MIMO State Space Representation}
\begin{align*}
    \dot{x} & =\underbrace{\overbrace{A}^{n\times n}\overbrace{x}^{n\times1}}_{n\times1}+\underbrace{\overbrace{B}^{n\times m}\overbrace{u}^{m\times1}}_{n\times1}             \\
    y\quad  & =\underbrace{\overbrace{C}^{\ell\times n}\overbrace{x}^{n\times1}}_{\ell\times1}+\underbrace{\overbrace{D}^{\ell\times m}\overbrace{u}^{m\times1}}_{\ell\times1}
\end{align*}
with
\begin{itemize}
    \item $x\in\mathbb{R}^n$, where $n$ is the order of the system.
    \item $u\in\mathbb{R}^m$, where $m$ is the number of inputs.
    \item $y\in\mathbb{R}^\ell$, where $\ell$ is the number of outputs.
\end{itemize}

\subsubsection{MIMO TF Matrix}

\ptitle{Remarks}
\begin{itemize}
    \item Diagonal TF matrices represent \textbf{decoupled} systems
    \item non-diagonal entries describe the of a certain input and a certain output
\end{itemize}

\ptitle{Continuous Time}

For an elementary input vector
\begin{equation*}
    u(t)=\underbrace{\begin{bmatrix}u_{1}  \\
            \vdots \\
            u_{m}
        \end{bmatrix}}_{direction} e^{st}
\end{equation*}
we get the output vector
\begin{equation*}
    y_{{\mathrm{ss}}}(t)=G(s)u(t)=
    \begin{bmatrix}G_{11}(s)    & \ldots & G_{1m}(s)     \\
               \vdots       & \ddots & \vdots        \\
               G_{\ell1}(s) & \ldots & G_{\ell m}(s)
    \end{bmatrix}
    \begin{bmatrix}u_{1}  \\
        \vdots \\
        u_{m}
    \end{bmatrix}e^{st}
\end{equation*}
where $G(s)\in\mathbb{C}^{\ell\times m}$ is a complex-valued matrix for $s\in \mathbb{C}$

\ptitle{Discrete Time}

For an elementary input vector
\begin{equation*}
    u[k]=\underbrace{\begin{bmatrix}u_{1}  \\
            \vdots \\
            u_{m}
        \end{bmatrix}}_{direction} z^k
\end{equation*}
we get the output vector
\begin{equation*}
    y_{{\mathrm{ss}}}[k]=G(z)u[k]=
    \begin{bmatrix}G_{11}(z)    & \ldots & G_{1m}(z)     \\
               \vdots       & \ddots & \vdots        \\
               G_{\ell1}(z) & \ldots & G_{\ell m}(z)
    \end{bmatrix}
    \begin{bmatrix}u_{1}  \\
        \vdots \\
        u_{m}
    \end{bmatrix}z^k
\end{equation*}
where $G(z)\in\mathbb{C}^{\ell\times m}$ is a complex-valued matrix for $z\in \mathbb{C}$

\paragraph{Interconnections}

\begin{itemize}
    \item matrix multiplication is non-commutative: $G_1G_2 \ne G_2G_1$
    \item one cannote divide by matrices: instead multiply by inverse
    \item to find the interconnection start at output $y$ and procceed against signal flow towards the input $u$
    \item series interconnection ($G_1$ closer to input): $y=G_2G_1u$
    \item feedback interconnection ($G_2$ in FB path, $(I+G_1G_2)$ invertible, $G_1G_2\ne -I$): $y={(I+G_1G_2)}^{-1}G_1u$
\end{itemize}
\subsection{MIMO Poles and Zeros}

\ptitle{SISO Poles and Zeros}

\begin{itemize}
    \item A pole with frequency $p$ is a generated frequency i.e. even if the excitation signal does not contain the frequency $p$ the output signal will contain this frequency as the initial condition response $x(t)=e^{At}x_0$ in modal form contains $p$
    \item A (transmission) zero with frequency $z$ is an absorbed frequency i.e. even if the excitation signal contains a term $e^{zt}$ it will not be visible in the output.
\end{itemize}

\subsubsection{MIMO Poles}
\begin{itemize}
    \item Also in MIMO system the poles are the eigenvalues of $A$.
    \item A MIMO system described by $\mathbf{G}(s)$ has a pole at $p$ if at least one of its entries has a pole at p.
    \item The characteristic polynomial of a minimal realization of $G(s)$ is the least common multiple (LCM) of the denominators of all possible minors of $G(s)$.
    \item The minors are all single entries of the $G(s)$ matrix as well as all possible square matrices (of all sizes!) in $G(s)$.
    \item The poles of $G(s)$ are the roots of the aforementioned LCM term.
\end{itemize}


\ptitle{MIMO Stability}

Same condition as in SISO sytems:
\begin{itemize}
    \item Stable if $Re(p_i \le 0)$ and the algebraic multiplicity of poles with zero real part is equal to their geometric multiplicity
    \item Asymptotically stable if $Re(p_i<0)$
\end{itemize}


\subsubsection{MIMO Zeros}
\begin{itemize}
    \item One can find MIMO zeros either from $\mathbf{G}(s)$ (transmission zeros) or from state space representation (invariant zeros).
    \item For minimal realizations they are the same.
    \item Similar to eigenvalues and their eigenvectors (as known from poles) we have zero frequencies an associated zero vectors.
\end{itemize}

\ptitle{Transmission Zeros}

$s = z_0$ is a transmission zero if there exists a non-zero, rational function $u_0(s)$ s.t.
\begin{equation*}
    \lim_{s\to z_0}\left[G(s)u_0(s)\right]e^{st}=0,\quad\forall t\geq0
\end{equation*}
which means that $\mathbf{G}(s)$ \textbf{loses rank} for $s\rightarrow z_0$.

\ptitle{Remarks}

\begin{itemize}
    \item A memo transmission zero is a pair of frequency $z0$ and direction $u_0(z_0)$.
    \item It can happen that a zero of $G$ is not visible in any of it's entries $G_{ij}(s)$.
    \item The direction of the zero decides whether a MIMO pole and zero at the same frequency \textbf{cancel} (i.e. there can be a pole and a zero at same frequency without cancellation!).
    \item A zero means not just one entry of $y$ is zero but that given an input vector $u \ne 0$, $y$ becomes the zero vector.
\end{itemize}

\paragraph{Invariant Zeros}

Using the system response to elementary inputs $x(t)=(sI-A)^{-1}Bu_0e^{st}$ and $y(t)=Cx_0e^{st}+Du_0e^{st}$ we must find the solution to
\begin{equation*}
    \begin{bmatrix}
        sI-A & -B \\
        C    & D
    \end{bmatrix}
    \begin{bmatrix}
        x_i \\
        u_i
    \end{bmatrix}=0
\end{equation*}


\ptitle{Remarks}

\begin{itemize}
    \item A non-trivial solution
          \begin{equation*}
              \begin{bmatrix}
                  x_i \\
                  u_i
              \end{bmatrix}=\mathbf{0}
          \end{equation*}
          means $\mathbf{y}=\mathbf{0}$ even though input and state are not zero!
    \item $s_i$ for which the aforementioned matrix becomes singular are called invariant zeros.
    \item An invariant zero is associated with a vector.
    \item Given zero initial condition and $u(t)=u_i e^{s_i t}$ (zero direction and zero frequency) we get
          \begin{equation*}
              y(t)=-Ce^{At}x_i
          \end{equation*} which is independent of the input and becomes $0$ for a corresponding non-zero initial condition.
    \item
\end{itemize}


\subsection{Realizations of MIMO Systems}
\subsubsection{Naive Realization}
Compute realization of each component of $G$ and assemble them to a new state space model.
\begin{align*}
    A & =\begin{bmatrix}
             A_{11} &        &        &        \\
                    & A_{12} &        &        \\
                    &        & \ddots &        \\
                    &        &        & A_{lm} \\
         \end{bmatrix},
    \quad B  =\begin{bmatrix}
                  b_{11} &        &        \\
                         & b_{12} &        \\
                         &        & \ddots \\
                  b_{21} &        &        \\
                         & \ddots &        \\
                         &        & b_{lm} \\
              \end{bmatrix}                                                  \\
    C & =\begin{bmatrix}
              & c_{11} & c_{12} & \dots &        &       &        &        &       &        \\
              &        &        &       & c_{21} & \dots &        &        &       &        \\
              &        &        &       &        &       & \ddots &        &       &        \\
              &        &        &       &        &       &        & c_{l1} & \dots & c_{lm} \\
         \end{bmatrix} \\
    D & = \begin{bmatrix}
              d_{11} & \dots  & d_{1m}   \\
              \vdots & \ddots & \vdots & \\
              d_{l1} & \dots  & d_{lm} & \\
          \end{bmatrix}
\end{align*}

\ptitle{Problems}

\begin{itemize}
    \item non-minimal
    \item redundant poles (sum of entrie's poles)
\end{itemize}

\subsubsection{Gilbert's Realization}

Given
\begin{equation*}
    G(s)=\frac{H(s)}{d(s)}+D
\end{equation*}
where
\begin{itemize}
    \item $d(s)$ is the least common denominator of all entries in $G(s)$
    \item $D=\lim_{s\to\infty}G(s)$ is the feed-through term.
\end{itemize}

Then, if $d(s)$ has no repeated roots, one can use \textbf{Gilbert's method}

\begin{enumerate}
    \item Write a (matrix) partial fraction expansion of $H(s)=d(s$)
          \begin{equation*}
              G(s)=\frac{R_1}{s-p_1}+\frac{R_2}{s-p_2}+\ldots+\frac{R_{n_d}}{s-p_{n_d}}+D
          \end{equation*} with the residues
          \begin{equation*}
              R_i=\lim_{s\to p_i}(s-p_i)G(s)
          \end{equation*}
          \begin{itemize}
              \item The ranks $r_i$ of $R_i$ indicate the number of poles at location $p_i$ that are needed for the realization.
              \item The order of the resulting state space model will be $n=\sum_{i=1}^{n_d}r_i\geq n_d$
          \end{itemize}
    \item Write $R_i=C_i B_i$ for some $B_i$, $C_i$ so that
          \begin{itemize}
              \item $B_i$ has $m$ (\# inputs) columns
              \item $C_i$ has $l$ (\# outputs) rows
              \item $B_i$ and $C_i$ have $r_i$ independent rows/columns
          \end{itemize}
    \item Assemble the state space model as
          \begin{align*}
              A & =\begin{bmatrix}
                       p_1 I_{r_1\times r_1} &                       &        &                                   \\
                                             & p_1 I_{r_2\times r_2} &        &                                   \\
                                             &                       & \ddots &                                   \\
                                             &                       &        & p_{n_d} I_{r_{n_d}\times r_{n_d}} \\
                   \end{bmatrix},\quad
              B =\begin{bmatrix}
                     B_1     \\
                     B_2     \\
                     \vdots  \\
                     B_{n_d} \\
                 \end{bmatrix}                                                                                                                             \\
              C & =\begin{bmatrix}
                       C_1 & C_2 & \dots & C_{n_d} \\
                   \end{bmatrix}, \quad
              D = D                                                                                                                                         \\
          \end{align*}
\end{enumerate}

\ptitle{Remarks}

\begin{itemize}
    \item Figures out the minimum number of ``copies'' of each pole that we need to construct a realization of a MIMO transfer function.
    \item For non-minimal realizations invariant zeros can give additional zeros (correspond to uncontrollable/unobservable modes)
\end{itemize}

\subsection{MIMO Signal Amplification}

\ptitle{Key Points}

\begin{itemize}
    \item Different input directions can generate wildly different outputs, even if they have the same entries (at different locations)
    \item Directionality depends on frequency as well
    \item We need ways to measure ``sizes'' in order to compare effects in different directions.
\end{itemize}

\subsubsection{Norms}

\ptitle{Properties}
\begin{align*}
     & \begin{Vmatrix}ax\end{Vmatrix}  =\begin{vmatrix}a\end{vmatrix}\begin{Vmatrix}x\end{Vmatrix},\quad\forall a\in\mathbb{R},x\in V & \text{(homogenity)}          \\
     & \|x+y\|                         \leq\|x\|+\|y\|                                                                                & \text{(triangle inequality)} \\
     & \left\|x\right\|                >0 \text{ for } x\neq0, \left\|x\right\|=0\leftrightarrow x=0                                  & \text{(positivity)}
\end{align*}

\ptitle{Example Norms}
\begin{itemize}
    \item Euclidian norm: $\left\|x\right\|=\sqrt{x^{\prime}x}$
    \item For Hermitian, pos. definite matrices: $\left\|x\right\|=\sqrt{x^{\prime}Qx}$
    \item p-norm: $\|x\|_p=\left(\sum_{i=1}^nx_i^p\right)^{1/p}$ with the important cases
          \begin{align*}
               & \|x\|_1=\sum_1^n\left|x_i\right|,               \\
               & \|x\|_2=\sqrt{\sum_{i=1}^nx_i^2}                \\
               & \left\|x\right\|_\infty=\max_i\left|x_i\right|.
          \end{align*}
\end{itemize}

\subsubsection{Matrix Norms}



Matrix norms are incuced by the norms of the vector spaces they operate on.

\ptitle{Properties of Induced Norms}


\ptitle{Example Norms}

