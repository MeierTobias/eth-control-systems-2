\section{MIMO Systems}

\subsection{State Space Representation}
\noindent\begin{align*}
    \underbrace{\dot{\mathbf{x}}}_{n\times 1} & = \underbrace{\mathbf{A}}_{n\times n}   \underbrace{\mathbf{x}}_{n\times1}+\underbrace{\mathbf{B}}_{n\times m}   \underbrace{\mathbf{u}}_{m\times1} \\
    \underbrace{\mathbf{y}}_{\ell\times 1}    & = \underbrace{\mathbf{C}}_{\ell\times n}\underbrace{\mathbf{x}}_{n\times1}+\underbrace{\mathbf{D}}_{\ell\times m}\underbrace{\mathbf{u}}_{m\times1}
\end{align*}
with
\begin{itemize}
    \item $\mathbf{x}\in\mathbb{R}^n$, where $n$ is the order of the system.
    \item $\mathbf{u}\in\mathbb{R}^m$, where $m$ is the number of inputs.
    \item $\mathbf{y}\in\mathbb{R}^\ell$, where $\ell$ is the number of outputs.
\end{itemize}

\subsection{TF Matrix}

\ptitle{Continuous Time}

For an elementary input vector $\mathbf{u}(t)$ we get the steady state output vector
\begin{equation*}
    \mathbf{y}_{{\mathrm{ss}}}(t)=\mathbf{G}(s)\mathbf{u}(t)=
    \begin{bmatrix}G_{11}(s)    & \ldots & G_{1m}(s)     \\
               \vdots       & \ddots & \vdots        \\
               G_{\ell1}(s) & \ldots & G_{\ell m}(s)
    \end{bmatrix}
    \underbrace{
    \begin{bmatrix}u_{1}  \\
        \vdots \\
        u_{m}
    \end{bmatrix}e^{st}}_{\mathbf{u}(t)}
\end{equation*}
where $\mathbf{G}(s)\in\mathbb{\mathbb{C}}^{\ell\times m}$ is a complex-valued matrix for $s\in \mathbb{C}$. The transfer function is then given by
\begin{equation*}
    \mathbf{G}(s)=\mathbf{C}\cdot{(s\cdot\mathbf{I}-\mathbf{A})}^{-1}\cdot \mathbf{B}+\mathbf{D}
\end{equation*}

\ptitle{Discrete Time}

The same holds for the discrete time case where $e^{st}$ is replaced by $z^k$ and $s$ by $z$. For further details see~\ref{disc:tf}.

\newpar{}
\ptitle{Remarks}
\begin{itemize}
    \item Diagonal TF matrices represent \textbf{decoupled} systems
    \item non-diagonal entries describe the relation of a certain input and a certain output
\end{itemize}

\subsubsection{Interconnections}

To find the interconnection start at output $\mathbf{y}$ and proceed against signal flow towards the input $\mathbf{u}$.

\newpar{}
\ptitle{Series Interconnection}
\newpar{}
\begin{center}
    \begin{tikzpicture}[auto, node distance=2cm,>=latex']
    \node [input, name=uinput]          (uinput)        {};
    \node [block, right of=uinput]      (G1)            {$\mathbf{G_1}$};
    \node [block, right of=G1]          (G2)            {$\mathbf{G_2}$};
    \node [output, right of=G2]         (output)        {};

    \draw [->] (rinput) -- node{$\mathbf{u}$} (G1);
    \draw [->] (G1) -- node{} (G2);
    \draw [->] (G2) -- node{$\mathbf{y}$} (output);
\end{tikzpicture}
\end{center}

\begin{equation*}
    \mathbf{y}=\mathbf{G_2G_1u}
\end{equation*}

\newpar{}
\ptitle{feedback interconnection}
\newpar{}
\begin{center}
    \begin{tikzpicture}[auto, node distance=1.5cm,>=latex']
    \node [input, name=uinput]          (uinput)        {};
    \node [sum, right of=uinput]        (sum1)          {};
    \node [block, right of=sum1]        (G1)            {$\mathbf{G_1}$};
    \node [block, below of=G1]          (G2)            {$\mathbf{G_2}$};
    \node [tmp, right of=G1]            (yt)            {};
    \node [output, right of=yt]         (output)        {};


    \draw [->] (rinput) -- node{$\mathbf{u}$} (sum1);
    \draw [->] (sum1) -- node{} (G1);
    \draw [->] (G1) -- node{$\mathbf{y}$} (output);
    \draw [->] (yt) |-(G2) -| node[pos=0.99] {$-$} (sum1);
\end{tikzpicture}
\end{center}

\begin{equation*}
    \mathbf{y}={(\mathbf{I}+\mathbf{G_1G_2})}^{-1}\mathbf{G_1u}
\end{equation*}


\subsection{Poles}

\begin{itemize}
    \item In MIMO systems the poles are the eigenvalues of $\mathbf{A}$.
    \item The \textbf{minors} are the determinants of all single entries of the $\mathbf{G}(s)$ matrix as well as the determinants of all possible square submatrices (of all sizes!) in $\mathbf{G}(s)$.
    \item The \textbf{characteristic polynomial} $d(s)$ of a minimal realization of $\mathbf{G}(s)$ is the least common multiple (LCM) of the denominators of all possible minors $G_{ij}$.
    \item The \textbf{poles} of $\mathbf{G}(s)$ are the roots of the aforementioned LCM term.
    \item Reminder: Even if the excitation signal does not contain frequency $p$ the output signal will contain this frequency as the initial condition response $x(t) = e^{\mathbf{A}t}x_0$ in modal form contains oscillations with frequency $p$.
\end{itemize}

% TODO: Example?

\newpar{}
\ptitle{Stability}

Same condition as in SISO systems:
\begin{itemize}
    \item \textbf{Stable} if $\mathrm{Re}(p_i \le 0)$ and the algebraic multiplicity of poles with zero real part is equal to their geometric multiplicity.
    \item \textbf{Asymptotically stable}\ if $\mathrm{Re}(p_i<0)$
\end{itemize}


\subsection{Zeros}
\begin{itemize}
    \item One can find MIMO zeros either from $\mathbf{G}(s)$ (transmission zeros) or from state space representation (invariant zeros).
    \item For minimal realizations they are the same.
          % \item Similar to eigenvalues and their eigenvectors (as known from poles) we have zero frequencies and associated zero vectors.
\end{itemize}

\subsubsection{Transmission Zeros}

$z_0$ is a transmission zero if there exists a non-zero, rational function $\mathbf{u}_0(s)$ such that
\begin{equation*}
    \lim_{s\to z_0}\left[\mathbf{G}(s)\mathbf{u}_0(s)\right]e^{st}=0,\quad\forall t\geq0
\end{equation*}
which means that $\mathbf{\mathbf{G}}(s)$ \textbf{loses rank} for $s\to z_0$.

\ptitle{Method to Determine Transmission Zeros}

\begin{enumerate}
    \item Calculate all minors (of all orders) of $\mathbf{G}(s)$
    \item Find the least common denominator $d_{\min}(s)$ of the minors
    \item Normalize highest order minors s.t.\ they have $d_{\min}(s)$ as denominator
    \item Find largest common numerator $n_{\max}(s)$ (usually product of $(s-z_i)$ terms) of the normalized minors from 3.
    \item The roots of $n_{\max}(s)$ are the MIMO zeros with their multiplicities
\end{enumerate}

\ptitle{Remarks}

\begin{itemize}
    \item A MIMO transmission zero is a pair of frequency $z_0$ and direction $\mathbf{u}_0(z_0)$.
    \item It can happen that a zero of $\mathbf{G}$ is not visible in any of it's entries $G_{ij}(s)$.
    \item The direction of the zero decides whether a MIMO pole and zero at the same frequency \textbf{cancel} (i.e.\ there can be a pole and a zero at same frequency without cancellation!).
    \item A zero means not just one entry of $\mathbf{y}$ is zero but that given an input vector $\mathbf{u} \ne \mathbf{0}$, $\mathbf{y}$ becomes the zero vector.
\end{itemize}

\begin{examplesection}[Example (From Script of Gioele Zardini)]
    Given TF matrix:
    \begin{equation*}
        \mathbf{G}(s)=\begin{pmatrix}
            \frac1{s+1} & \frac1{s+2}           & \frac{2\cdot(s+1)}{(s+2)\cdot(s+3)} \\
            0           & \frac{s+3}{{(s+1)}^2} & \frac{s+4}{s+1}
        \end{pmatrix}
    \end{equation*}
    1. Minors (already simplifed determinants):
    \begin{align*}
        \frac{1}{s+1},\frac{1}{s+2},\frac{2\cdot(s+1)}{(s+2)\cdot(s+3)},0,\frac{s+3}{{(s+1)}^{2}},\frac{s+4}{s+1} & \text{ (1st order)} \\
        \frac{s+3}{{(s+1)}^3},\frac{1}{s+1},-\frac{s+4}{{(s+1)}^2}                                                & \text{ (2nd order)}
    \end{align*}
    2. Find $d_{\min}(s)$:
    \begin{equation*}
        {(s+1)}^3\cdot(s+2)\cdot(s+3)
    \end{equation*}
    3. Normalize highest order (here 2nd order) minors:
    \begin{align*}
         & \frac{{(s+3)}^2\cdot(s+2)}{{(s+1)}^3\cdot(s+2)\cdot(s+3)}                  \\
         & \frac{{(s+1)}^2\cdot(s+2)\cdot(s+3)}{{(s+1)}^3\cdot(s+2)\cdot(s+3)}        \\
         & -\frac{(s+4)\cdot(s+1)\cdot(s+2)\cdot(s+3)}{{(s+1)}^3\cdot(s+2)\cdot(s+3)}
    \end{align*}
    4. Find $n_{\max}(s)$:
    \begin{equation*}
        n_{\max}(s)=(s+3)(s+2)
    \end{equation*}
    5. Identify zeros: $z_1=-3,\;z_2=-2$
\end{examplesection}

\subsubsection{Invariant Zeros}

Invariant zeros are the values of $s$ for which the following matrix becomes singular and therefor the determinant of the matrix is zero:
\begin{equation*}
    \det \left(\begin{bmatrix}
            s\mathbf{I}-\mathbf{A} & -\mathbf{B} \\
            \mathbf{C}             & \mathbf{D}
        \end{bmatrix}\right) \overset{!}{=}0
\end{equation*}
The corresponding input vectors can then be determined by plugging the invariant zeros into the equation below:
\begin{equation*}
    \begin{bmatrix}
        s\mathbf{I}-\mathbf{A} & -\mathbf{B} \\
        \mathbf{C}             & \mathbf{D}
    \end{bmatrix}
    \begin{bmatrix}
        \mathbf{x}_i \\
        \mathbf{u}_i
    \end{bmatrix}=\mathbf{0}
\end{equation*}

\ptitle{Remarks}

\begin{itemize}
    \item An invariant zero $z_i$ is associated with a vector $\mathbf{u}_i$.
    \item Given zero initial condition and $\mathbf{u}(t)=\mathbf{u}_i e^{s_i t}$ (zero direction and zero frequency) we get
          \begin{equation*}
              \mathbf{y}(t)=-\mathbf{C}e^{\mathbf{A}t}\mathbf{x_i}
          \end{equation*} which is independent of the input and becomes $0$ for a corresponding non-zero initial condition.
    \item For non-minimal realizations invariant zeros can give additional zeros (correspond to uncontrollable/unobservable modes).
\end{itemize}


\subsection{Realizations of MIMO Systems}
\subsubsection{Naive Realization}
Compute realization of each component of $\mathbf{G}(s)$ and assemble them to a new state space model.
\begin{align*}
    \mathbf{A} & =\begin{bmatrix}
                      \mathbf{A}_{11} &                 &        &                 \\
                                      & \mathbf{A}_{12} &        &                 \\
                                      &                 & \ddots &                 \\
                                      &                 &        & \mathbf{A}_{lm} \\
                  \end{bmatrix},
    \quad \mathbf{B}  =\begin{bmatrix}
                           \mathbf{b}_{11} &                 &                 \\
                                           & \mathbf{b}_{12} &                 \\
                                           &                 & \ddots          \\
                           \mathbf{b}_{21} &                 &                 \\
                                           & \ddots          &                 \\
                                           &                 & \mathbf{b}_{lm} \\
                       \end{bmatrix}                                                                    \\
    \mathbf{C} & =\begin{bmatrix}
                       & \mathbf{c}_{11} & \mathbf{c}_{12} & \dots &                 &       &        &                 &       &                 \\
                       &                 &                 &       & \mathbf{c}_{21} & \dots &        &                 &       &                 \\
                       &                 &                 &       &                 &       & \ddots &                 &       &                 \\
                       &                 &                 &       &                 &       &        & \mathbf{c}_{l1} & \dots & \mathbf{c}_{lm} \\
                  \end{bmatrix} \\
    \mathbf{D} & = \begin{bmatrix}
                       \mathbf{d}_{11} & \dots  & \mathbf{d}_{1m} \\
                       \vdots          & \ddots & \vdots          \\
                       \mathbf{d}_{l1} & \dots  & \mathbf{d}_{lm} \\
                   \end{bmatrix}
\end{align*}

\ptitle{Problems}

\begin{itemize}
    \item non-minimal
    \item redundant poles (sum of entrie's poles)
\end{itemize}

\subsubsection{Gilbert's Realization}

Given
\begin{equation*}
    \mathbf{G}(s)=\frac{\mathbf{H}(s)}{d(s)}+\mathbf{D}
\end{equation*}
where
\begin{itemize}
    \item $d(s)$ is the least common denominator of all entries in $\mathbf{G}(s)$
    \item $\mathbf{D}=\lim_{s\to\infty}\mathbf{G}(s)$ is the feed-through term.
\end{itemize}

\newpar{}
Then, if $d(s)$ has no repeated roots, one can use \textbf{Gilbert's method}:

\newpar{}
\begin{enumerate}
    \item Write a (matrix) partial fraction expansion of $\frac{\mathbf{H}(s)}{d(s)}$
          \begin{equation*}
              \mathbf{G}(s)=\frac{\mathbf{R_1}}{s-p_1}+\frac{\mathbf{R_2}}{s-p_2}+\ldots+\frac{\mathbf{R_{n_d}}}{s-p_{n_d}}+\mathbf{D}
          \end{equation*} with the residues
          \begin{equation*}
              \mathbf{R_i}=\lim_{s\to p_i}(s-p_i)\mathbf{G}(s)
          \end{equation*}
          \begin{itemize}
              \item The ranks $r_i$ of $\mathbf{R_i}$ indicate the number of poles at location $p_i$ that are needed for the realization.
              \item The order of the resulting state space model will be $n=\sum_{i=1}^{n_d}r_i\geq n_d$
              \item Ensure that $d(s)$ has no repeated roots (otherwise one has to use the generalized Gilbert's method).
              \item Check $n\le n_d$ after calculating $r_i$.
          \end{itemize}
    \item Write $\overbrace{\mathbf{R_i}}^{l\times m}=\mathbf{\overbrace{\mathbf{C}_i}^{l\times r_i}\overbrace{ \mathbf{B}_i}^{r_i \times m}}$ for some $\mathbf{B_i}$, $\mathbf{C_i}$ so that
          \begin{itemize}
              \item $\mathbf{B_i}$ has $m$ (\# inputs) columns
              \item $\mathbf{C_i}$ has $l$ (\# outputs) rows
              \item $\mathbf{B_i}$ and $\mathbf{C_i}$ have $r_i$ independent rows/columns
          \end{itemize}
    \item Assemble the state space model as
          \begin{gather*}
              \mathbf{A}  =\begin{bmatrix}
                  p_1 \mathbf{I}_{r_1\times r_1} &                                &        &                                            \\
                                                 & p_2 \mathbf{I}_{r_2\times r_2} &        &                                            \\
                                                 &                                & \ddots &                                            \\
                                                 &                                &        & p_{n_d} \mathbf{I}_{r_{n_d}\times r_{n_d}} \\
              \end{bmatrix} \\
              \mathbf{B}  =\begin{bmatrix}
                  \mathbf{B}_1     \\
                  \mathbf{B}_2     \\
                  \vdots           \\
                  \mathbf{B}_{n_d} \\
              \end{bmatrix}, \quad
              \mathbf{C}  =\begin{bmatrix}
                  \mathbf{C}_1 & \mathbf{C}_2 & \dots & \mathbf{C}_{n_d} \\
              \end{bmatrix}, \quad
              \mathbf{D} = \mathbf{D}
          \end{gather*}
\end{enumerate}

\ptitle{Remarks}

\begin{itemize}
    \item Figures out the minimum number of ``copies'' of each pole that we need to construct a realization of a MIMO transfer function.
    \item $\mathbf{A}$ simply contains diagonal matrices of dimension $r_i \times r_i$ with $p_i$ on their diagonals.
\end{itemize}

% TODO: Add an example?
% \begin{examplesection}[Example of Gilbert's Realization]
%     Given the TF
%     \begin{equation*}
%         \mathbf{G}(s) = \begin{bmatrix}
%             1 & \frac{1}{s+1} \\
%             0 & 1
%         \end{bmatrix}
%     \end{equation*}
%     The LCM denominator $d(s)$ is
%     \begin{equation*}
%         d(s) = s+1
%     \end{equation*}
%     and the feed-through term is
%     \begin{equation*}
%         \mathbf{D} = \mathbf{I}
%     \end{equation*}
%     
% \end{examplesection}

% \subsection{Signal Amplification}

\subsection{Singular Value Decomposition (SVD)}

Any matrix $\mathbf{A}\in \mathbb{C}^{m\times n}$ can be decomposed as
\noindent\begin{equation*}
    \mathbf{A}=\mathbf{U}\Sigma \mathbf{V}^{\mathsf{H}}\quad \begin{cases}
        \mathbf{U}\in \mathbb{C}^{m\times m} & \text{left singular vectors}                          \\
        \Sigma\in \mathbb{C}^{m\times n}     & \text{singular values }\sigma_i \text{, desc.\ order} \\\
        \mathbf{V}\in \mathbb{C}^{n\times n} & \text{right singular vectors}\
    \end{cases}
\end{equation*}

\ptitle{$2\times 3$ Example}

\noindent\begin{equation*}
    \mathbf{A}=
    \underbrace{\begin{bmatrix}
            \vert        & \vert        \\
            \mathbf{u}_1 & \mathbf{u}_2 \\
            \vert        & \vert
        \end{bmatrix}}_{\textsf{Rotation}}
    \underbrace{\begin{bmatrix}
            \sigma_1 & 0        & 0 \\
            0        & \sigma_2 & 0
        \end{bmatrix}}_{\textsf{Scaling \& Dimensions}}
    \underbrace{\begin{bmatrix}
            \text{---} & \mathbf{v_1} & \text{---} \\
            \text{---} & \mathbf{v_2} & \text{---} \\
            \text{---} & \mathbf{v_3} & \text{---}
        \end{bmatrix}}_{\textsf{Rotation}}
\end{equation*}
with eigenvalues
\noindent\begin{gather*}
    \lambda_{u_1}=\lambda_{v_1}={\sigma_1}^2\\
    \lambda_{u_2}=\lambda_{v_2}={\sigma_2}^2\\
    \lambda_{v_3}=0
\end{gather*}


\subsubsection{Procedure}
Assuming $\mathbf{A}$ has ``full'' rank = $\min(m,n)$:
\begin{enumerate}
    \item calculate eigenvalues (start with smaller matrix, pad with 0)
          \noindent\begin{align*}
               & \lambda_{\mathbf{V}}\in \mathbb{R}^n: & \det\left(\lambda \mathbf{I}-\mathbf{A}^{\mathsf{H}}\mathbf{A}\right)\overset{!}{=}0 \\
               & \lambda_{\mathbf{U}}\in \mathbb{R}^m: & \det\left(\lambda \mathbf{I}-\mathbf{AA}^{\mathsf{H}}\right)\overset{!}{=}0
          \end{align*}
    \item calculate eigenvectors of $\mathbf{U},\mathbf{V}$
          \noindent\begin{align*}
               & \mathbf{V}:                & \left(\lambda_{\mathbf{v}_i} \mathbf{I} -\mathbf{A}^{\mathsf{H}} \mathbf{A}\right)\mathbf{x} & \overset{!}{=}\mathbf{0}          \\
               & \mathbf{U}:                & \mathbf{u}_i                                                                                 & = \frac{1}{\sigma_i}\mathbf{Av}_i \\
               & \mathrm{if}~\sigma_i = 0 : & \left(\lambda_{\mathbf{u}_i} \mathbf{I} -\mathbf{AA}^{\mathsf{H}}\right)\mathbf{x}           & \overset{!}{=}\mathbf{0}
          \end{align*}
          where $\lambda_{\mathbf{u}i}=0$ if $\mathbf{V}$ has smaller dimensions than $\mathbf{U}$ and vice versa.
    \item calculate singular values $\sigma\in \mathbb{R}^{\min(m,n)}$ and compose $\Sigma$
          \noindent\begin{align*}
              \sigma_i & = \sqrt{\lambda_{u_i}} = \sqrt{\lambda_{v_i}}\quad i=0,1,\cdots, \min(m,n) \\
              \Sigma   & = \begin{bmatrix}
                               \sigma_1 & \cdots & 0          & \mathbf{0} \\
                               \vdots   & \ddots & \vdots     & \mathbf{0} \\
                               0        & \cdots & \sigma_{m} & \mathbf{0}
                           \end{bmatrix}\text{ or }
              \begin{bmatrix}
                  \sigma_1   & \cdots & 0          \\
                  \vdots     & \ddots & \vdots     \\
                  0          & \cdots & \sigma_{n} \\
                  \mathbf{0} & \cdots & \mathbf{0}
              \end{bmatrix}
          \end{align*}
    \item \noindent\begin{equation*}
              \mathbf{A}=\mathbf{U}\Sigma \mathbf{V}^{\mathsf{H}}
          \end{equation*}
\end{enumerate}

\ptitle{Remarks}

\begin{itemize}
    \item Write out the dimensions of $\mathbf{U}$, $\Sigma$ and $\mathbf{V}$ directly
    \item One can normalize the columns of $\mathbf{U}$ and $\mathbf{V}$ separately
\end{itemize}

\subsubsection{Interpetation}
When looking at the columns of $\mathbf{A}$ separatly
\noindent\begin{equation*}
    \mathbf{Av}_i=\sigma_i \mathbf{u}_i
\end{equation*}
shows that a input in direction $\mathbf{v}_i$ results in an output in direction $\mathbf{u}_i$ with amplification $\sigma_i$.

\newpar{}
\ptitle{Zeros and Poles}

If the matrix $\mathbf{A}$ contains $s$, the directions of poles and zeros can be found with SVD:
\begin{enumerate}
    \item Set $s=z, p+0.0001$
    \item Calculate SVD of $\mathbf{A}(s)$
    \item Evaluate singular values:
          \begin{enumerate}
              \item The \textbf{input zero direction} is the column $\mathbf{v_i}$ with corresponding $\sigma_i=0$.
              \item The output zero direction is the column $\mathbf{u_i}$ with corresponding $\sigma_i=0$.
              \item The \textbf{input pole direction} is the column $\mathbf{v_i}$ with largest $\sigma_i$.
              \item The output pole direction is the column $\mathbf{u_i}$ with largest $\sigma_i$.
          \end{enumerate}
\end{enumerate}

\newpar{}
\ptitle{Condition Number}
\noindent\begin{equation*}
    \kappa(\mathbf{A})=\frac{\sigma_{\max}(\mathbf{A})}{\sigma_{\min}(\mathbf{A})}
\end{equation*}

\textbf{Remarks}:
\begin{itemize}
    \item A high condition number suggests that the system is \textit{strongly directional} i.e.\ \textit{ill-conditioned} and thus difficult to control.
    \item Often, considering only the \textit{strong} singular values is sufficient.
\end{itemize}

\subsection{Norms}
% \ptitle{Usage of Norms}
\begin{itemize}
    \item Inputs vectors with same but permuted entries can generate wildly different outputs due to directionality.
    \item Directionality depends on frequency as well.
    \item Norms are a way to measure ``sizes'' in order to compare effects in different directions.
\end{itemize}

\ptitle{Properties}
\begin{align*}
     & \left\| a\mathbf{x}\right\|  =\left|a\right|\left\|\mathbf{x}\right\|,\quad\forall a\in\mathbb{R},\mathbf{x}\in V                 & \text{(homogeneity)}         \\
     & \left\|\mathbf{x}+\mathbf{y}\right\|                         \leq\|\mathbf{x}\|+\|\mathbf{y}\|                                    & \text{(triangle inequality)} \\
     & \left\|\mathbf{x}\right\|                >0 \text{ for } \mathbf{x}\neq0, \left\|\mathbf{x}\right\|=0\leftrightarrow \mathbf{x}=0 & \text{(positivity)}
\end{align*}

\ptitle{Example Norms}
\begin{itemize}
    \item Euclidean norm: $\left\|\mathbf{x}\right\|=\sqrt{\mathbf{x}^{\mathsf{H}}\mathbf{x}}$
    \item For Hermitian, pos.\ definite matrices: $\left\|\mathbf{x}\right\|=\sqrt{\mathbf{x}^{\mathsf{H}} Qx}$
    \item p-norm: $\|\mathbf{x}\|_p{=\left(\sum_{i=1}^n x_i^p\right)}^{1/p}$ with the important cases
          \begin{align*}
               & \|\mathbf{x}\|_1=\sum_1^n\left|x_i\right|,               \\
               & \|\mathbf{x}\|_2=\sqrt{\sum_{i=1}^n {x_i}^2}             \\
               & \left\|\mathbf{x}\right\|_\infty=\max_i\left|x_i\right|.
          \end{align*}
\end{itemize}

\subsubsection{Matrix Norms}
A $m\times n$ complex matrix can be seen as an operator between the vector spaces $\mathbb{C}^n$ and $\mathbb{C}^m$ i.e.
\begin{equation*}
    \mathbf{A}^{m\times n}:\mathbb{C}^n\to\mathbb{C}^m,\mathbf{x}\mapsto \mathbf{A}x
\end{equation*}
If we then provide the vector spaces with a norm, a \textbf{matrix norm is induced} by the norms of the vector spaces it operates on.

\begin{equation*}
    \|\mathbf{A}\|_p:=\sup_{\mathbf{x}\neq0}\frac{\|\mathbf{Ax}\|_p}{\|\mathbf{x}\|_p}=\max_{\|\mathbf{x}\|_p=1}\|\mathbf{Ax}\|_p
\end{equation*}
which means that the induced p-norm of $\mathbf{A}$ measures how much multiplication by $\mathbf{A}$ amplifies the p-norm of a vector. In other words, $\|\mathbf{A}\|_p$ is the \textbf{gain} of the operator $\mathbf{A}$.

\newpar{}
\ptitle{Properties of Induced Norms}

As for vectors, homogeneity, triangle inequality and positivity hold for matrix norms too. Additionally we have:
\begin{equation*}
    \left\|\mathbf{Ax}\right\|_p\leq\left\|a\right\|_p\left\|\mathbf{x}\right\|_p
\end{equation*}
and the submultiplicative property:
\begin{align*}
    \|\mathbf{AB}\|_p                                     & \leq\|\mathbf{A}\|_p\|\mathbf{B}\|_p                                                      \\
    \|\mathbf{AB} \mathbf{x}\|_p                          & \leq\|\mathbf{A}\|_p\|\mathbf{Bx}\|_p\leq\|\mathbf{A}\|_p\|\mathbf{B}\|_p\|\mathbf{x}\|_p \\
    \frac{\|\mathbf{AB} \mathbf{x}\|_p}{\|\mathbf{x}\|_p} & \leq\|\mathbf{A}\|_p\|\mathbf{B}\|_p
\end{align*}


\paragraph{Specific Norms}

\ptitle{Induced $p$-norms}

For the $p = 1$ norm we get the \textbf{max-sum column}:
\begin{equation*}
    \|\mathbf{A}\|_1=\max_j\sum_{i=1}^m|a_{ij}|,\qquad x_k=\begin{cases}
        1 & k=j         \\
        0 & \text{else}
    \end{cases}
\end{equation*}

\newpar{}
For the $p = 2$ norm we get the \textbf{maximal amplification}:
\noindent\begin{equation*}
    \|\mathbf{A}\|_{2,\mathrm{ind}}:=\sup_{x\neq0}\frac{\|\mathbf{Ax}\|_2}{\|\mathbf{x}\|_2}=\sigma_{\max}(\mathbf{A}),\qquad \mathbf{x}=\mathbf{v}_1
\end{equation*}
and \textbf{minimal amplification} (assuming rank($\mathbf{A}$)=$n$):
\noindent\begin{equation*}
    \inf_{x\neq0}\frac{\|\mathbf{Ax}\|_2}{\|\mathbf{x}\|_2}=\sigma_n(\mathbf{A}),\qquad \mathbf{x}=\mathbf{v}_n
\end{equation*}

\newpar{}
For the $p = \infty$ norm we get the \textbf{max-sum row}:
\begin{equation*}
    \|\mathbf{A}\|_\infty=\max_i\sum_{j=1}^n|a_{ij}|\qquad x_k=\pm 1
\end{equation*}

\ptitle{Frobenius Norm}

\noindent\begin{align*}
    \|\mathbf{A}\|_F: & ={\left(\sum_{i=1}^m\sum_{j=1}^n|a_{ij}|^2\right)}^{1/2}=\text{Trace}{(\mathbf{A}^{\mathsf{H}}\mathbf{A})}^{1/2} \\
                      & = {\left(\sum_{i=1}^r\sigma_i{(\mathbf{A})}^2\right)}^{1/2}
\end{align*}

\begin{itemize}
    \item non-induced
    \item still has submultiplicative property
\end{itemize}

\textbf{Remarks}: $x_k$ describes the elements of the vector $\mathbf{x}$ for which the norm-specific maximum amplification happens.

\subsubsection{Signal Norms}
Signals are functions and functions are vectors.
\noindent\begin{equation*}
    w:\mathbb{T}\rightarrow \mathbb{R}^n, \quad w(t)=(w_1(t), w_2(t), \ldots w_n(t))
\end{equation*}
\ptitle{p-norms}

\textbf{1-norm}: Action
\noindent\begin{equation*}
    \|w\|_1=\begin{cases}
        \sum\limits_{k\in\mathbb{Z}}\|w[k]\|_1 & (\mathsf{DT}) \\
        \int_{-\infty}^\infty\|w(t)\|_1 dt     & (\mathsf{CT})
    \end{cases}
\end{equation*}
\newpar{}
\textbf{2-norm:} Square of the Energy
\noindent\begin{equation*}
    \|w\|_2^2=
    \begin{cases}
        \sum\limits_{k\in\mathbb{Z}}w[k]'w[k] = \sum\limits_{k\in\mathbb{Z}}\|w[k]\|_2^2 & \mathsf{(DT)} \\
        \int_{-\infty}^\infty w(t)'w(t) dt = \int_{-\infty}^\infty\|w(t)\|_2^2 dt        & \mathsf{(CT)}
    \end{cases}
\end{equation*}

\newpar{}
\textbf{$\infty$-norm:} Peak Magnitude
\noindent\begin{equation*}
    \|w\|_\infty=\sup_{t\in\mathbb{T}}\|w(t)\|_\infty=\sup_{t\in\mathbb{T}}\max_{i=1\ldots n}|w_i(t)|
\end{equation*}
where the absolute value of the signal must be taken because the maximum could be negative.

\ptitle{Power} (not a norm)
\noindent\begin{equation*}
    \mathrm{pow}{(w)}^2=
    \begin{cases}
        \lim\limits_{N\to+\infty}\frac{1}{2N}\sum\limits_{k=-N}^N\|w[k]\|_2^2 & \mathsf{(DT)} \\
        \lim\limits_{T\to+\infty}\frac{1}{2T}\int_{-T}^T\|w(t)\|_2^2 dt       & \mathsf{(CT)}
    \end{cases}
\end{equation*}

\paragraph{Relationships}
\noindent\begin{align*}
    \|w\|_2<\infty                                         & \Rightarrow \mathrm{pow}{(w)}=0                         \\
    \mathsf{DT}:\;\;\|w\|_2<\infty                         & \Rightarrow \|w\|_\infty<\infty                         \\
    \mathsf{CT}:\;\;\|w\|_2<\infty                         & \;\cancel{\Leftrightarrow}\; \|w\|_\infty<\infty        \\
    (\mathrm{pow}{(w)}<\infty) \land (\|w\|_\infty<\infty) & \Rightarrow \mathrm{pow}{(w)} < \|w\|_\infty^2          \\
    (\|w\|_1<\infty) \land (\|w\|_\infty<\infty)           & \Rightarrow \|w\|_2<\sqrt{\|w\|_1\|w\|_\infty} < \infty
\end{align*}

\begin{center}
    \includegraphics[width = 0.7\linewidth]{signal_norm.png}
\end{center}

\newpar{}
\subsubsection{System Norms}
A system maps an input $u$ to an output $y$:
\noindent\begin{equation*}
    y=Su
\end{equation*}

The induced norm is defined as
\noindent\begin{equation*}
    \|S\|_{p,\mathrm{ind}}:=\sup_{u\neq0}\frac{\|Su\|_p}{\|u\|_p}=\sup_{u\neq0}\frac{\|y\|_p}{\|u\|_p}
\end{equation*}

\newpar{}
\paragraph{Parseval's Identity}
\noindent\begin{equation*}
    \underbrace{\|y\|_{\mathcal{L}_2}^2}_{\textsf{Time}}=\int_0^\infty\|y(t)\|_2^2 dt=\frac1{2\pi}\int_{-\infty}^{+\infty}\|Y(j\omega)\|_2^2 d\omega=\underbrace{\|Y(s)\|_{\mathcal{H}_2}^2}_{\textsf{Freq.} (s=j\omega)}
\end{equation*}

\paragraph[L2-induced Norm]{$\mathcal{L}_2$-induced Norm}
\noindent\begin{equation*}
    \sup_{u\neq0}\frac{\|y\|_{\mathcal{L}_2}}{\|u\|_{\mathcal{L}_2}}=\sup_{\omega\in\mathbb{R}}\sigma_{\max}[G(j\omega)]=:\|G\|_{\mathcal{H}_\infty}
\end{equation*}
\textbf{Remarks}:
\begin{itemize}
    \item In SISO systems, the $\mathcal{H}_\infty$ norm corresponds to the peak value of the Bode plot.
    \item The submultiplicative property for $\mathcal{H}_\infty$ holds.
\end{itemize}

\paragraph[H2-induced Norm]{$\mathcal{H}_2$-norm}

The $\mathcal{H}_2$-norm measures
\begin{enumerate}
    \item The energy of the impulse response:
          \noindent\begin{equation*}
              \|g\|_{\mathcal{L}_2}^2:= \|G\|_{\mathcal{H}_2}^2
          \end{equation*}
    \item The energy of the response to initial conditions of the form $x(0)=\mathbf{B}u_0$ for $u_0={(1,1,\ldots, 1)}^{\mathsf{T}}$
    \item The (expected) power of the response to white noise:
          \noindent\begin{equation*}
              \mathbb{E}\left[\lim_{T\to+\infty}\frac{1}{T}\text{Tr}\left(\int_0^T y(t){y(t)}^{\mathsf{H}} dt\right)\right] =  \|g\|_{\mathcal{L}_2}^2:= \|G\|_{\mathcal{H}_2}^2
          \end{equation*}
\end{enumerate}
\textbf{Remarks}:
\begin{itemize}
    \item The LQR/LQE problem can be though as a $\mathcal{H}_2$ minimization problem.
    \item In order for $\|g\|_{\mathcal{L}_2}^2:= \|G\|_{\mathcal{H}_2}^2<\infty$, the system needs to be strictly causal, i.e. $\lim_{\omega\to\infty}G(j\omega)=0$
    \item The submultiplicative property for $\mathcal{H}_\infty$ does not hold.
\end{itemize}