\section{Introduction to MIMO Systems}
\subsection{Fundamentals of MIMO Systems}

\ptitle{Comparison to SISO}
MIMO Systems:
\begin{itemize}
    \item have TF matrices
    \item impose the concept of directionality
    \item can adapt many SISO ideas but not all (poles, zeros, system response phase measurement)
\end{itemize}

\subsubsection{MIMO State Space Representation}
\begin{align*}
    \dot{x} & =\underbrace{\overbrace{A}^{n\times n}\overbrace{x}^{n\times1}}_{n\times1}+\underbrace{\overbrace{B}^{n\times m}\overbrace{u}^{m\times1}}_{n\times1}             \\
    y\quad  & =\underbrace{\overbrace{C}^{\ell\times n}\overbrace{x}^{n\times1}}_{\ell\times1}+\underbrace{\overbrace{D}^{\ell\times m}\overbrace{u}^{m\times1}}_{\ell\times1}
\end{align*}
with
\begin{itemize}
    \item $x\in\mathbb{R}^n$, where $n$ is the order of the system.
    \item $u\in\mathbb{R}^m$, where $m$ is the number of inputs.
    \item $y\in\mathbb{R}^\ell$, where $\ell$ is the number of outputs.
\end{itemize}

\subsubsection{MIMO TF Matrix}

\ptitle{Remarks}
\begin{itemize}
    \item Diagonal TF matrices represent \textbf{decoupled} systems
    \item non-diagonal entries describe the of a certain input and a certain output
\end{itemize}

\ptitle{Continuous Time}

For an elementary input vector
\begin{equation*}
    u(t)=\underbrace{\begin{bmatrix}u_{1}  \\
            \vdots \\
            u_{m}
        \end{bmatrix}}_{direction} e^{st}
\end{equation*}
we get the output vector
\begin{equation*}
    y_{{\mathrm{ss}}}(t)=G(s)u(t)=
    \begin{bmatrix}G_{11}(s)    & \ldots & G_{1m}(s)     \\
               \vdots       & \ddots & \vdots        \\
               G_{\ell1}(s) & \ldots & G_{\ell m}(s)
    \end{bmatrix}
    \begin{bmatrix}u_{1}  \\
        \vdots \\
        u_{m}
    \end{bmatrix}e^{st}
\end{equation*}
where $G(s)\in\mathbb{C}^{\ell\times m}$ is a complex-valued matrix for $s\in \mathbb{C}$

\ptitle{Discrete Time}

For an elementary input vector
\begin{equation*}
    u[k]=\underbrace{\begin{bmatrix}u_{1}  \\
            \vdots \\
            u_{m}
        \end{bmatrix}}_{direction} z^k
\end{equation*}
we get the output vector
\begin{equation*}
    y_{{\mathrm{ss}}}[k]=G(z)u[k]=
    \begin{bmatrix}G_{11}(z)    & \ldots & G_{1m}(z)     \\
               \vdots       & \ddots & \vdots        \\
               G_{\ell1}(z) & \ldots & G_{\ell m}(z)
    \end{bmatrix}
    \begin{bmatrix}u_{1}  \\
        \vdots \\
        u_{m}
    \end{bmatrix}z^k
\end{equation*}
where $G(z)\in\mathbb{C}^{\ell\times m}$ is a complex-valued matrix for $z\in \mathbb{C}$

\paragraph{Interconnections}

\begin{itemize}
    \item matrix multiplication is non-commutative: $G_1G_2 \ne G_2G_1$
    \item one cannote divide by matrices: instead multiply by inverse
    \item to find the interconnection start at output $y$ and procceed against signal flow towards the input $u$
    \item series interconnection ($G_1$ closer to input): $y=G_2G_1u$
    \item feedback interconnection ($G_2$ in FB path, $(I+G_1G_2)$ invertible, $G_1G_2\ne -I$): $y={(I+G_1G_2)}^{-1}G_1u$
\end{itemize}
\subsection{MIMO Poles and Zeros}
\subsubsection{MIMO Poles}
\paragraph{MIMO Stability}

\subsubsection{MIMO Zeros}
One can find MIMO zeros either from ...
\paragraph{Invariant Zeros}

\subsection{Realizations of MIMO Systems}
\subsubsection{Naive Realization}
\subsubsection{Gilbert's Realization}


\subsection{MIMO Signal Amplification}
\subsubsection{Norms}

\ptitle{Properties}

\ptitle{Example Norms}

\subsubsection{Matrix Norms}
Matrix norms are incuced by the norms of the vector spaces they operate on.

\ptitle{Properties of Induced Norms}


\ptitle{Example Norms}

