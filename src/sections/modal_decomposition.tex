\section{Modal Decomposition}

\subsection{Recall: Eigendecomposition}
Let $\lambda_i$ and $\mathbf{v}_i$ be an eigenvalue and an eigenvector of $\mathbf{A}$.
\begin{align*}
    \mathbf{Av}_i & = \lambda_i \mathbf{v}_i                                   \\
    \mathbf{A} \begin{bmatrix}
                   \vdots       &        & \vdots       \\
                   \mathbf{v}_1 & \cdots & \mathbf{v}_n \\
                   \vdots       &        & \vdots
               \end{bmatrix}
                  & = \begin{bmatrix}
                          \vdots                 &        & \vdots                 \\
                          \lambda_1 \mathbf{v}_1 & \cdots & \lambda_n \mathbf{v}_n \\
                          \vdots                 &        & \vdots
                      \end{bmatrix} \\
                  & = \underbrace{\begin{bmatrix}
                                          \vdots       &        & \vdots       \\
                                          \mathbf{v}_1 & \cdots & \mathbf{v}_n \\
                                          \vdots       &        & \vdots
                                      \end{bmatrix}}_{V}
    \underbrace{\begin{bmatrix}
                        \lambda_1 & 0      & 0         \\
                        0         & \ddots & 0         \\
                        0         & 0      & \lambda_n
                    \end{bmatrix}}_{\Lambda}                                 \\
    \mathbf{AV}   & = \mathbf{V}\bm{\Lambda}                                   \\
    \bm{\Lambda}  & = \mathbf{V}^{-1}\mathbf{AV}
\end{align*}
As shown $\bm{\Lambda}$ is a diagonal matrix containing the eigenvalues of $\mathbf{A}$ and $\mathbf{V}$ is a column stack of the eigenvectors.

\subsection{Similarity Transformation}
The state space representation is not unique. This can be used to transform the model into a more \flqq{}useful/readable\frqq{} representation.
\begin{align*}
    \tilde{\mathbf{x}} & = \mathbf{Px}       \\
    \tilde{\mathbf{A}} & = \mathbf{PAP}^{-1} \\
    \tilde{\mathbf{B}} & = \mathbf{PB}       \\
    \tilde{\mathbf{C}} & = \mathbf{CP}^{-1}  \\
    \tilde{\mathbf{D}} & = \mathbf{D}
\end{align*}
If $\mathbf{P}$ is set to the inverse of the eigenvector matrix of $\mathbf{A}$ we get:
\begin{align*}
    \mathbf{P}         & = \mathbf{V}^{-1}                           \\
    \tilde{\mathbf{x}} & = \mathbf{V}^{-1}\mathbf{x}                 \\
    \tilde{\mathbf{A}} & = \mathbf{V}^{-1}\mathbf{AV} = \bm{\Lambda} \\
    \tilde{\mathbf{B}} & = \mathbf{V}^{-1}\mathbf{B}                 \\
    \tilde{\mathbf{C}} & = \mathbf{CV}                               \\
    \tilde{\mathbf{D}} & = \mathbf{D}
\end{align*}
Where $\tilde{\mathbf{A}}$ is a diagonal matrix consisting of the eigenvalues of $\mathbf{A}$.

\subsection{Modal Time Response}
\subsubsection{Modal Coordinates}
\begin{itemize}
    \item The eigenvector $\mathbf{v}_i$ defines the shape of the $i$-th mode.
    \item The modal coordinate $\tilde{\mathbf{x}}_i=\mathbf{V}^{-1}\mathbf{x}$ scales the mode (e.g., at the initial condition)
    \item The eigenvalue $\lambda_i$ defines how the amplitude of the mode evolves over time.

          \begin{tabularx}{\linewidth}{@{}p{0.5\linewidth}X@{}}
              \quad Real $\lambda_i$                            & \textrightarrow{} $e^{\lambda_i t}x_0$                         \\
              \quad Complex $\lambda_i = \sigma_i + j \omega_i$ & \textrightarrow{} $e^{\sigma_i t}\sin(\omega_i t + \phi_0)x_0$ \\
              \quad Repeated $\lambda_i$                        & \textrightarrow{} $t^p e^{\lambda_i t}$
          \end{tabularx}
\end{itemize}

\begin{examplesection}[Example: Interpretation of Modal Decomposition]
    The modal decomposition gives valuable insight regarding the behavior of a system. The modes (eigenvalues and corresponding eigenvectors) are analyzed one at a time.
    \begin{enumerate}
        \item The magnitude of the eigenvalues describe the influence of the mode.
        \item The property (real, complex or repeated) of the eigenvalue describes the behavior over time of that mode.
        \item The corresponding eigenvector describes the states/dimensions the mode is acting on. For example $x,y,z$ or $\varphi, \dot{\varphi}$
    \end{enumerate}
\end{examplesection}

\subsubsection{Homogeneous Response}
For a given initial condition $x(0)=x_0$ the homogeneous response can be computed as follows:\\
\ptitle{In Modal Coordinates}
\begin{align*}
    \tilde{\mathbf{x}}(t) & =e^{\tilde{\mathbf{A}}t}\tilde{\mathbf{x}}(0) \\
    \tilde{x}_i(t)        & =e^{\lambda_{i}t}\tilde{x}_i(0)
\end{align*}
which means that each mode evolves independently (of other modes) over time.\\
\ptitle{In Standard Coordinates}
\begin{equation*}
    \mathbf{x}(t)=\sum_{i=1}^{n}e^{\lambda_{i}t}\tilde{x}_i(0)\mathbf{v}_i
\end{equation*}
because for an initial condition $x_0 = v_i$ the time response is given by
\begin{equation*}
    \mathbf{x}(t)=e^{\mathbf{A}t}\mathbf{v}_i=e^{\lambda_{i}t}\mathbf{v}_i
\end{equation*}
and $\mathbf{v}_i$ form a basis of the corresponding state space.

\begin{examplesection}[Example: Simple Pendulum]
    For a simple pendulum with no damping and no input the state model is given by:
    \begin{equation*}
        \dot{\mathbf{x}}=\underbrace{\begin{bmatrix}
                0            & 1 \\
                -\frac{g}{l} & 0
            \end{bmatrix}}_{\mathbf{A}}\mathbf{x}
    \end{equation*}
    where
    \begin{equation*}
        \mathbf{x}=\begin{bmatrix}
            \varphi \\
            \dot{\varphi}
        \end{bmatrix}
    \end{equation*}

    The eigenvalues and eigenvectors of $\mathbf{A}$ are
    \begin{equation*}
        \lambda_{1,2} = \pm j \sqrt{\frac{g}{l}} \qquad \mathbf{v}_{1,2}=\begin{bmatrix}
            1 \\
            \pm j \sqrt{\frac{g}{l}}
        \end{bmatrix}
    \end{equation*}

    With the initial condition
    \noindent\begin{equation*}
        \mathbf{V}^{-1}\mathbf{x}(0) = \mathbf{V}^{-1}{[1, 0]}^T = {\left[\frac{1}{2}, \frac{1}{2}\right]}^T = \tilde{\mathbf{x}}(0)
    \end{equation*}
    the eigenvectors can be visualized as following:

    \includegraphics[width=\linewidth]{images/mode_shapes_pendulum.png}

    Due to the purely complex eigenvalues, the eigenvectors only rotate when evolving over time.

    Combining the two modes, we get
    \begin{align*}
        \mathbf{x}_{ICR}(t) & = e^{j\sqrt{\frac{g}{l}}t}\frac{1}{2}
        \begin{bmatrix}
            1 \\
            \textcolor{red}{+j \sqrt{\frac{g}{l}}}
        \end{bmatrix} + e^{-j\sqrt{\frac{g}{l}}t}\frac{1}{2}
        \begin{bmatrix}
            1 \\
            \textcolor{red}{-j \sqrt{\frac{g}{l}}}
        \end{bmatrix}                      \\
        \mathbf{x}_{ICR}(t) & = \begin{bmatrix}
                                    \varphi \\
                                    \textcolor{red}{\dot{\varphi}}
                                \end{bmatrix}
        = \begin{bmatrix}
              \cos(\omega t) \\
              \textcolor{red}{-\omega\sin(\omega t)}
          \end{bmatrix}
    \end{align*}
\end{examplesection}
