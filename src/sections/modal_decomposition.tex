\section{Modal Decomposition}

\subsection{Recall: Eigendecomposition}
Let $\lambda_i$ and $v_i$ be an eigenvalue and an eigenvector of $A$.
\begin{align*}
    A \vec{v_i} & = \lambda_i \vec{v_i}                    \\
    A \begin{bmatrix}
          \vdots &        & \vdots \\
          v_1    & \cdots & v_n    \\
          \vdots &        & \vdots
      \end{bmatrix}
                & = \begin{bmatrix}
                        \vdots        &        & \vdots        \\
                        \lambda_1 v_1 & \cdots & \lambda_n v_n \\
                        \vdots        &        & \vdots
                    \end{bmatrix} \\
                & = \underbrace{\begin{bmatrix}
                                        \vdots &        & \vdots \\
                                        v_1    & \cdots & v_n    \\
                                        \vdots &        & \vdots
                                    \end{bmatrix}}_{V}
    \underbrace{\begin{bmatrix}
                        \lambda_1 & 0      & 0         \\
                        0         & \ddots & 0         \\
                        0         & 0      & \lambda_n
                    \end{bmatrix}}_{\Lambda}             \\
    AV          & = V\Lambda                               \\
    \Lambda     & = V^{-1}AV
\end{align*}
As shown $\Lambda$ is a diagonal matrix containing the eigenvalues of $A$ and $V$ is a column stack of the eigenvectors.

\subsection{Similarity Transformation}
The state space representation is not unique. This can be used to transform the model into a more \flqq{}useful/readable\frqq{} representation.
\begin{align*}
    \tilde{x} & = Px       \\
    \tilde{A} & = PAP^{-1} \\
    \tilde{B} & = PB       \\
    \tilde{C} & = CP^{-1}  \\
    \tilde{D} & = D
\end{align*}
If $P$ is set to the inverse of the eigenvector matrix of $A$ we get:
\begin{align*}
    P         & = V^{-1}             \\
    \tilde{x} & = V^{-1}x            \\
    \tilde{A} & = V^{-1}AV = \Lambda \\
    \tilde{B} & = V^{-1}B            \\
    \tilde{C} & = CV                 \\
    \tilde{D} & = D
\end{align*}
Where $\tilde{A}$ is a diagonal matrix consisting of the eigenvalues of $A$.

\subsection{Modal coordinates}
\begin{itemize}
    \item The eigenvector $v_i$ defines the shape of the $i$-th mode.
    \item The modal coordinate $\tilde{x}_i=V^{-1}x$ scales the mode (e.g., at the initial condition)
    \item The eigenvalue $\lambda_i$ defines how the amplitude of the mode evolves over time.

          \begin{tabularx}{\linewidth}{@{}p{0.5\linewidth}X@{}}
              \quad Real $\lambda_i$                            & \textrightarrow{} $e^{\lambda_i t}x_0$                         \\
              \quad Complex $\lambda_i = \sigma_i + j \omega_i$ & \textrightarrow{} $e^{\sigma_i t}\sin(\omega_i t + \phi_0)x_0$ \\
              \quad Repeated $\lambda_i$                        & \textrightarrow{} $t^p e^{\lambda_i t}$
          \end{tabularx}
\end{itemize}

\begin{examplesection}[Example: Simple Pendulum]
    For a simple pendulum with no damping and no input the state model is given by:
    \begin{equation*}
        \dot{x}=\underbrace{\begin{bmatrix}
                0            & 1 \\
                -\frac{g}{l} & 0
            \end{bmatrix}}_{A}x
    \end{equation*}
    where
    \begin{equation*}
        x=\begin{bmatrix}
            \varphi \\
            \dot{\varphi}
        \end{bmatrix}
    \end{equation*}

    The eigenvalues and eigenvectors of $A$ are
    \begin{equation*}
        \lambda_{1,2} = \pm j \sqrt{\frac{g}{l}} \qquad v_{1,2}=\begin{bmatrix}
            1 \\
            \pm j \sqrt{\frac{g}{l}}
        \end{bmatrix}
    \end{equation*}

    With the initial condition $x(0) = {[1, 0]}^T$ the eigenvectors can be visualized as following:

    \includegraphics[width=\linewidth]{images/mode_shapes_pendulum.png}

    Due to the purely complex eigenvalues, the eigenvectors only rotate when evolving over time.

    Combining the two modes, we get
    \begin{align*}
        x(t) = \begin{bmatrix}
                   \varphi \\
                   \textcolor{red}{\dot{\varphi}}
               \end{bmatrix}
        = \begin{bmatrix}
              \cos(\omega t) \\
              \textcolor{red}{-\omega\sin(\omega t)}
          \end{bmatrix}
    \end{align*}
\end{examplesection}