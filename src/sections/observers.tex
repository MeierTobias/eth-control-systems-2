\section{Observers}

In case we desire full-state feedback but there are no sensors available to measure each state, we could try to simulate the system with its estimated states $\hat{\mathbf{x}}$.

\begin{center}
    \includegraphics[width=0.8\linewidth]{images/state_estimation.png}
\end{center}

\begin{itemize}
    \item we choose $\hat{\mathbf{x}}_0=\mathbf{0}$
    \item the control input $u$ is computed from the simulated state
    \item both the simulated model and the physical plant receive the same control input
\end{itemize}

\newpar{}
\ptitle{Estimation Error Dynamics}

\begin{itemize}
    \item the estimation error $\boldsymbol{\eta}:=\mathbf{x}-\hat{\mathbf{x}}$ shares the dynamics of the OL system $\dot{\boldsymbol{\eta}}=\mathbf{A}\boldsymbol{\eta}$.
    \item this can lead to undesirable estimation error dynamics (slow convergence, oscillations, divergence in case of unstable OL system)
\end{itemize}

\subsection{The Luenberger Observer}
We don't have to fully rely on our state estimation but can instead, use the available sensory information to improve it.
\begin{itemize}
    \item the signal $\mathbf{y}-\hat{\mathbf{y}}=\mathbf{C}(\mathbf{x}-\hat{\mathbf{x}}) = \mathbf{C}\boldsymbol{\eta}$ is called \textbf{innovation} and is the input to the Luenberger \textbf{observer gain} $\mathbf{L}$
    \item the Luenberger observer corrects the estimated state by a linear feedback on the innovation
    \item in other words we design a controller to get a good estimate of the physical plant's state
\end{itemize}

\begin{center}
    \includegraphics[width=0.8\linewidth]{images/Luenberger.png}
\end{center}

\newpar{}

\ptitle{Estimation Error Dynamics}

The states evolve as follows:
\begin{align*}
    \dot{\mathbf{x}}       & =\mathbf{Ax}+\mathbf{Bu}                                                         \\
    \dot{\hat{\textbf{x}}} & =\mathbf{A}\hat{\mathbf{x}}+\mathbf{Bu}+\mathbf{LC}(\mathbf{x}-\hat{\mathbf{x}}) \\
    \dot{\hat{\textbf{x}}} & =(\mathbf{A-LC})\hat{\mathbf{x}}+\mathbf{Bu}+\mathbf{Ly}
\end{align*}
Hence,
\begin{equation*}
    \dot{\eta}=\dot{\mathbf{x}}-\dot{\hat{\textbf{x}}}=\mathbf{A}(\mathbf{x}-\hat{\mathbf{x}})-\mathbf{LC}(\mathbf{x}-\hat{\mathbf{x}})=(\mathbf{A}-\mathbf{LC})\boldsymbol{\eta}
\end{equation*}

\ptitle{Remarks:}

\begin{itemize}
    \item The system has to be \textbf{observable}.
    \item The observer dynamics are given by $\mathbf{A}-\mathbf{LC}$.
\end{itemize}

\subsubsection{Observer Pole Placement}
For state feedback we obtained the CL eigenvalues $\mathbf{A}-\mathbf{BK}$.
For the Luenberger observer we can state the \textbf{dual} problem, namely placing the poles of
\begin{equation*}
    \mathbf{A}-\mathbf{LC}
\end{equation*}
respectively
\begin{equation*}
    \mathbf{A}^{\mathsf{T}}-\mathbf{C}^{\mathsf{T}} \mathbf{L}^{\mathsf{T}}\text{ (same Eigenvalues)}
\end{equation*}
using the methods mentioned in Section~\ref{sec:pole_placement} (Pole Placement) or LQE (see~\ref{obs::LQE}).

\newpar{}
\ptitle{Observer Pole Placement}

$\mathbf{L}={\left[\ell_0,\cdots,\ell_{n-1}\right]}^{\mathsf{T}}$ can be obtained by comparing the eigenvalues of the observer with the desired ones:
\noindent\begin{equation*}
    \det\bigl(\lambda \mathbf{I} -(\mathbf{A}-\mathbf{LC})\bigr) \overset{!}{=} \varphi_{\mathrm{cl,des}}(\lambda)
\end{equation*}

\ptitle{Ackermann Observer Design}

Similarly to state feedback we get
\begin{align*}
    \mathbf{L}                            & =\varphi_{\mathrm{cl,des}}(\mathbf{A})\mathbf{O}^{-1}\begin{bmatrix}
                                                                                                     0, & \ldots, & 0, & 1
                                                                                                 \end{bmatrix}^{\mathsf{T}} \\
    \varphi_{\mathrm{cl,des}}(s)          & =s^n+\alpha_{n-1}s^{n-1}+\ldots+\alpha_0=(s-\lambda_1)\ldots(s-\lambda_n)       \\
    \varphi_{\mathrm{cl,des}}(\mathbf{A}) & =\mathbf{A}^n+\alpha_{n-1}\mathbf{A}^{n-1}+\ldots+\alpha_0 \mathbf{I}           \\
                                          & = (\mathbf{A}-\lambda_1 \mathbf{I})\ldots(\mathbf{A}-\lambda_n \mathbf{I})
\end{align*}

\textbf{Remarks:}
\begin{itemize}
    \item $\mathbf{O}$ has to be invertible. This is the case for observable SISO LTI systems.
    \item \textbf{observability} is a \textbf{necessary and sufficient} condition for arbitrary pole placement
    \item \textbf{detectability} is \textbf{necessary} for stable observer dynamics (as we need to observe the unstable poles)
    \item If the system is
          \begin{itemize}
              \item not observable (so at least one $c_k=0$)
              \item diagonalizable and given in modal coordinates
          \end{itemize}
          \begin{align*}
              \dot{\boldsymbol{\eta}_k} & =\lambda_k\boldsymbol{\eta}_k-\mathbf{I}_k\sum_{j=1}^k\left(\tilde{c}_j \boldsymbol{\eta}_j\right)
          \end{align*}
          then the unobservable poles cannot be moved (cannot assign these poles)
\end{itemize}

\newpar{}

\ptitle{Noise Sensitivity}

\begin{itemize}
    \item Large values for $\mathbf{L}$ make our measurements more \textbf{sensitive to noise}.
    \item Hence, we need to find a trade-off between estimation dynamics and robustness to noise.
\end{itemize}

\subsection{Noise Models}
We model noise as a stochastic process with a \textbf{noise signal}
\begin{equation*}
    \mathbf{w}:t\mapsto \mathbf{w}(t)\in\mathbb{R}^n
\end{equation*}
where each $\mathbf{w}(t)$ is the realization of a random variable.

\subsubsection{White Noise}
``Perfect'' but unphysical (values change instantaneously) noise can be modeled as follows:
\begin{enumerate}
    \item $\mathbb{E}[\mathbf{w}(t)]=0$:\newline
          the signal has zero mean.
    \item $\mathbb{E}[\mathbf{w}(t){\mathbf{w}(t)}^{\mathsf{T}}]=\mathbf{W}$:\newline
          the signal value at each time instance has covariance matrix $\mathbf{W}$, for some positive definite matrix $\mathbf{W}$.
    \item $\mathbb{E}[\mathbf{w}(t){\mathbf{w}(\tau)}^{\mathsf{T}}]=0$:\newline
          the signal values at two different times are not correlated. In fact, we will assume that signal values at two different times are \textbf{independent} i.e.\ the matrix $\mathbf{W}$ is diagonal
\end{enumerate}
If such a signal additionally is Gaussian-distributed it is called ``zero mean Gaussian white noise''.

\subsubsection{Colored Noise}
To simulate physical noise we let a white noise input $w$ propagate through an LTI system model (filter) and use the system's output as noise signal:
\begin{align*}
    \dot{\mathbf{x}} & =\mathbf{Ax}+\mathbf{Bw} \\
    \mathbf{ y}      & = \mathbf{Cx}
\end{align*}
The LTI system usually acts as low-pass filter in this case.


\subsection{Linear Quadratic Estimator (LQE) / Kalman Filter}\label{obs::LQE}

Noise affects our state estimation adversely. The LQE / Kalman filter aims at an optimal trade-off between
\begin{itemize}
    \item trusting the sensory information and hence, having higher observer gains in $\mathbf{L}$ (increases noise-sensitivity, left image)
    \item trusting the state-space model and hence, having lower observer gains (slower convergence) in $\mathbf{L}$ (decreases noise-sensitivity, right image)
\end{itemize}
and corrects state estimation based on available measurements and assumptions on noise.
\begin{center}
    \includegraphics[width=\linewidth]{images/trade_off_noise.png}
\end{center}

\textbf{Remark}: LQE can be applied only to \textbf{LTI systems}.\

\subsubsection{Optimal LQE Design}
For the LQE problem, we get the following LTI Model
\begin{align*}
    \dot{\mathbf{x}}(t) & = \mathbf{Ax}(t)+\mathbf{Bu}(t)+\mathbf{w}(t) \\
    \mathbf{y}(t)       & = \mathbf{Cx}(t)+\mathbf{n}(t)
\end{align*}
with the \textbf{Gaussian white noise} signals
\begin{itemize}
    \item $\mathbf{w}(t)$: process noise
    \item $\mathbf{n}(t)$: sensor noise
\end{itemize}
We model the errors with the covariance matrices
\begin{align*}
    \mathbf{Q} & =\mathbb{E}[\mathbf{w}(t){\mathbf{w}(t)}^{\mathsf{T}}]  &                & \text{ (process)} \\
    \mathbf{R} & =\mathbb{E}[\mathbf{n}(t){\mathbf{n}(t)}^{\mathsf{T}}], & \forall t\geq0 & \text{ (sensor)}
\end{align*}
The (Luenberger) observer is given by
\begin{align*}
    \dot{\hat{\textbf{x}}}(t) & =(\mathbf{A}-\mathbf{LC})\hat{\mathbf{x}}(t)+\mathbf{Bu}(t)+\mathbf{Ly}(t) \\
    \hat{\mathbf{y}}(t)       & =\mathbf{C}\hat{\mathbf{x}}(t)
\end{align*}

\ptitle{Minimization Problem}

The steady-state covariance of the state estimation error
\begin{equation*}
    \lim_{t\to+\infty}\mathbb{E}\left[(\mathbf{x}(t)-\hat{\mathbf{x}}(t)){(\mathbf{x}(t)-\hat{\mathbf{x}}(t))}^{\mathsf{T}}\right]
\end{equation*}
is minimized by solving the Riccati equation (\textit{ARE})
\begin{equation*}
    \mathbf{AY}+\mathbf{YA}^{\mathsf{T}}-\mathbf{YC}^{\mathsf{T}} \mathbf{R}^{-1}\mathbf{CY}+\mathbf{Q}=\mathbf{0}
\end{equation*}
for the \textbf{positive definite} matrix $\mathbf{Y}$ and choosing the \textit{optimal estimation gain} $\mathbf{L}$ as
\begin{equation*}
    \mathbf{L}^{\mathsf{T}}=\mathbf{R}^{-1}\mathbf{CY}
\end{equation*}

\ptitle{Remarks}:

\begin{itemize}
    \item the system has to be detectable (check for observability) and ($\mathbf{A,Q}$) stabilizable (check for reachability, treat $\mathbf{Q}$ like $\mathbf{B}$)\newline
          (e.g. $n=2: \mathbf{\mathcal{R}}= \left[\mathbf{Q}\quad \mathbf{AQ}\right]$)
    \item $\mathbf{Y}$ is \textit{real}, \textit{symmetric}, \textit{positive definite} ($\mathrm{Re}(\lambda_i) > 0, \mathrm{Im}(\lambda_i) = 0$) and has to fulfil the \textit{Sylvester criterion} e.g.\ for $n=2$:
          \noindent\begin{equation*}
              \mathbf{Y}=\begin{bmatrix}
                  a & b \\
                  b & d
              \end{bmatrix}; \quad a>0, \quad ad-b^2>0
          \end{equation*}
    \item for symmetric $\mathbf{Y}$ one has that $\mathbf{Y}\mathbf{A}^{\mathsf{T}}={(\mathbf{AY})}^{\mathsf{T}}$
    \item as expected from duality, $\mathbf{L}$ is the transpose of $\mathbf{K}$, obtained for the pair $(\mathbf{A}^{\mathsf{T}}, \mathbf{C}^{\mathsf{T}})$, and for weight matrices $\mathbf{Q}$ and $\mathbf{R}$.
    \item in practice, we can measure the noise to get a first estimate of $\mathbf{Q}$, $\mathbf{R}$
    \item as a guideline, one should make the innovations $\mathbf{C}\boldsymbol{\eta}$ as white as possible
\end{itemize}
