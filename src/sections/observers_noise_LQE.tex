\section{Observers}
\subsection{Luenberger Observer}
Assuming we desire full-state feedback but there are no sensors available to measure each state we could try to simulate the system (with it's states).
\subsubsection{State Estimation Without Observer}
\begin{itemize}
    \item we choose $\hat{x_0}=0$
    \item the control input $u$ is computed from the simulated state
    \item both the simulated model and the physical plant receive the same control input
\end{itemize}
\begin{center}
    \includegraphics[width=0.8\linewidth]{images/state_feedback.png}\\
\end{center}
\ptitle{Estimation Error Dynamics}

\begin{itemize}
    \item the estimation error $\mathbf{\eta}:=\mathbf{x}-\mathbf{\hat{x}}$ shares the dynamics of the OL system $\mathbf{\dot{\eta}}=\mathbf{A\eta}$!
    \item this can lead to undesirable estimation error dynamics (slow convergence, oscillations, divergence in case of unstable OL system)
    \item control basically dissapears for the estimation error as we try to control the plant
\end{itemize}

\subsubsection{The Luenberger Observer}
We don't have to fully rely on our state estimation but can instead use the available sensory information to improve it.
\begin{itemize}
    \item the signal $y-\hat{y}=\mathbf{C}(x-\hat{x})$ is called \textbf{innovation} and is the input to the Luenberger \textbf{observer gain} $\mathbf{L}$
    \item the Luenberger Observer corrects the estimated state by a linear feedback on the innovation
    \item in other words we design a controller to get a good estimate of the physical plant's state
\end{itemize}
\begin{center}
    \includegraphics[width=0.8\linewidth]{images/Luenberger.png}
\end{center}
\ptitle{Estimation Error Dynamics}

The states evolve as follows:
\begin{align*}
    \mathbf{x}^+       & =\mathbf{Ax}+\mathbf{Bu}                     \\
    \mathbf{\hat{x}^+} & =\mathbf{A}\hat{x}+\mathbf{Bu}+\mathbf{LC(x-\hat{x})}
\end{align*}
Hence,
\begin{equation*}
    \mathbf{\eta^+}=\mathbf{x^+}-\mathbf{\hat{x}^+}=\mathbf{A}(\mathbf{x}-\mathbf{\hat{x}})-\mathbf{LC}(\mathbf{x}-\mathbf{\hat{x}})=(\mathbf{A}-\mathbf{LC})\mathbf{\eta}
\end{equation*}

\subsubsection{Observer Pole Placement}

\ptitle{Ackermann Observer Design}
%


\ptitle{Observer Pole Placement for Unobservable Systems}
%


\subsection{Noise Models}
\subsubsection{White Noise}

\subsubsection{Colored Noise}



\subsection{Linear Quadratic Estimation (LQE) Kalman Filter}
Note: this method can be applied only to LTI systems.

\subsubsection{Optimal LQE Design}
For the LQE problem we get the following LTI Models

\ptitle{Remarks}
%
\begin{itemize}
    \item a
\end{itemize}